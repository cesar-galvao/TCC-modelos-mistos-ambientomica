% Please add the following required packages to your document preamble:
% \usepackage{longtable}
% Note: It may be necessary to compile the document several times to get a multi-page table to line up properly


% Structure of the metadata related to field trials included in the Embrapa Rice Breeding Dataset.

\section{Dicionário de variáveis do ERBD relacionadas aos ensaios}\label{sec: dicionario-erbd1}

\begin{longtable}{@{} p{4cm} p{4cm} p{8cm} @{}} 
	\caption{Dicionário de variáveis do ERBD relacionadas aos ensaios}
	\toprule
	\textbf{Variável} & \textbf{Nome}                    & \textbf{Detalhes}                                                           \\* \midrule
	\endfirsthead
	%
	\endhead
	%
	\bottomrule
	\endfoot
	%
	\endlastfoot
	%
	TRIAL                & Código do ensaio        & String único que identifica o ensaio                               \\
	SYST   & Sistema de cultivo           & Indica tanto o subprograma de melhoramento quanto o ambiente do ensaio. Níveis: Irrigado ou de Sequeiro            \\
	YEAR                 & Ano do ensaio           & Ano de preparação do ensaio. Ex: 2005: temporada 2005/2006         \\
	DATE                 & Data de plantio         & Dia de plantio de sementes secas. Formato DD/MM/AAAA               \\
	ST                   & Estado do Brasil        & Estado do Brasil onde o ensaio foi conduzido                       \\
	LOCATION             & Local de plantio        & Nome do município onde o ensaio foi conduzido                      \\
	LOC                  & Local de plantio        & Termo abreviado que indica o município                             \\
	TYPE   & Tipo de ensaio               & Tipo de ensaio. ER: Ensaios Regionais de Rendimento; VCU: Valor de Cultivo e Uso (Ensaios Avançados de Rendimento) \\
	DESIGN & Desenho Experimental         & O desenho estatístico do ensaio. RCB: delineamento de blocos completos ao acaso; LAT: delineamento em látice       \\
	MEAN   & Média de rendimento de grãos & Média geral do ensaio do rendimento de grãos (kg ha$^{-1}$)                                       \\
	H\textasciicircum{}2 & Hereditariedade         & Hereditariedade de sentido amplo do rendimento de grãos            \\
	CV                   & Coeficiente de Variação & Coeficiente de variação experimental para rendimento de grãos (\%) \\* \bottomrule
\end{longtable}