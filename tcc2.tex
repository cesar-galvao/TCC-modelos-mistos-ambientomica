\documentclass[12pt, a4paper, twoside]{report}
\usepackage[left = 3cm, top = 3cm, right = 2cm, bottom = 2cm]{geometry}
\usepackage[brazilian]{babel}
\usepackage[T1]{fontenc} %brazilian pt hyphenation
\usepackage[utf8]{inputenc}
\usepackage{amsmath, amsfonts, amssymb}
\numberwithin{equation}{subsection} %subsection
\usepackage{fancyhdr}
\usepackage{graphicx}
\usepackage{colortbl}
\usepackage{titletoc,titlesec}
\usepackage{setspace}
\usepackage{indentfirst}
\usepackage{natbib}
\usepackage[colorlinks=true, allcolors=black]{hyperref}
%\usepackage[brazilian,hyperpageref]{backref}
%\usepackage[alf]{abntex2cite}
\usepackage{multirow} % https://www.ctan.org/pkg/multirow
\usepackage{float} % https://www.ctan.org/pkg/float
\usepackage{booktabs} % https://www.ctan.org/pkg/booktabs
\usepackage{enumitem} % https://www.ctan.org/pkg/enumitem
\usepackage{quoting} % https://www.ctan.org/pkg/quoting
\usepackage{epigraph}
\usepackage{anyfontsize}
\usepackage{caption}
\usepackage{adjustbox}
\usepackage{bm}
%\usepackage{tocbibind}
\usepackage[titletoc,page]{appendix}
\usepackage{url,hyperref}
\usepackage{fancyhdr}
\usepackage{booktabs}
\usepackage{amssymb}
\usepackage{textcomp}
\usepackage{tikz}
\usepackage{bbm}
\usepackage{subfig}
\usepackage{graphicx}
\usepackage{rotating}
\usepackage{afterpage}
\usepackage{makecell} % Para usar \thead
\usepackage{tabularx} % Para usar o ambiente tabularx
\usepackage{adjustbox} % Para ajustar a largura da tabela
\usepackage{pdflscape} % Para usar o ambiente landscape
\usepackage{caption} % Para personalizar a legenda
\renewcommand{\arraystretch}{1.5} % Ajusta o espaçamento entre as linhas
\usepackage{float}
\usepackage{longtable}




%\usepackage[style=abnt]{biblatex}
%\bibliographystyle{plainnat}
\raggedbottom % https://latexref.xyz/_005craggedbottom.html

\newtheorem{teo}{Teorema}[section]
\newtheorem{lema}[teo]{Lema}
\newtheorem{cor}[teo]{Corolário}
\newtheorem{prop}[teo]{Proposição}
\newtheorem{defi}{Definição}
\newtheorem{exem}{Exemplo}

\newcommand{\titulo}{Modelos mistos aplicados à recomendação de cultivares no contexto da ambientômica}
\newcommand{\autor}{César Augusto Galvão}
\newcommand{\orientador}{ Prof. Dr. Leandro T. Correia}
\newcommand{\coorientador}{ Prof(a). Dr. Rafael T. Tassinari}

\pagestyle{fancy}
\fancyhf{}
%\renewcommand{\headrulewidth}{0pt}
\setlength{\headheight}{16pt}
%C - Centro, L - Esquerda, R - Direita, O - impar, E - par
\fancyhead[RO, LE]{\thepage}
\renewcommand{\sectionmark}[1]{\markboth{#1}{}}

\titlecontents{section}[0cm]{}{\bf\thecontentslabel\ }{}{\titlerule*[.75pc]{.}\contentspage}
\titlecontents{subsection}[0.75cm]{}{\thecontentslabel\ }{}{\titlerule*[.75pc]{.}\contentspage}

\setcounter{secnumdepth}{3}
%\setcounter{tocdepth}{2}

\DeclareCaptionFormat{myformat}{\centering \fontsize{10}{12}\selectfont#1#2#3}
\captionsetup{format=myformat}


\begin{document}

\begin{titlepage}
\begin{center}
\begin{figure}[h!]
	\centering
		\includegraphics[scale = 0.8]{unb.png}
	\label{fig:unb}
\end{figure}
{\bf Universidade de Brasília \\
\bf Departamento de Estatística}
\vspace{5cm}

\setcounter{page}{0}
\null
\textbf{\titulo}
\vspace{2.5cm}


\vspace{0.2cm}
\textbf{\autor}
\end{center}
\vspace{1.5cm}

\begin{flushright}
\begin{minipage}{7.5cm}
\parbox[t]{7.5cm}{Relatório apresentado para o Departamento de Estatística da Universidade de Brasília como parte dos requisitos necessários para obtenção do grau de Bacharel em Estatística.}
\end{minipage}
\end{flushright}

\vspace{5cm}

\begin{center}
{\bf{Brasília} \\ }
\bf{2024}
\end{center}

\end{titlepage}

\thispagestyle{empty}

\begin{center}
\textbf{\autor} \\
\vspace{5cm}
\textbf{\titulo} \\
\vspace{3cm}
\small
Orientador: \orientador \\
Coorientador(a): \coorientador
\end{center}


\vspace*{3cm}

\begin{flushright}
\begin{minipage}{7.5cm}
 \parbox[t]{7.5cm}{Relatório apresentado para o Departamento de Estatística da Universidade de Brasília como parte dos requisitos necessários para obtenção do grau de Bacharel em Estatística.}
\end{minipage}
\end{flushright}

\vspace{5cm}

\begin{center}
{\bf{Brasília} \\ }
\bf{2024}
\end{center}


\pagenumbering{gobble}
%\setcounter{page}{2}
\onehalfspacing




\setlength{\parindent}{1.5cm}
\setlength{\parskip}{0.2cm}
\setlength{\intextsep}{0.5cm}

\titlespacing*{\section}{0cm}{0cm}{0.5cm}
\titlespacing*{\subsection}{0cm}{0.5cm}{0.5cm}
\titlespacing*{\subsubsection}{0cm}{0.5cm}{0.5cm}
\titlespacing*{\paragraph}{0cm}{0.5cm}{0.5cm}

\titleformat{\paragraph}
{\normalfont\normalsize\bfseries}{\theparagraph}{1em}{}
%\titlespacing*{\paragraph}
%{0pt}{3.25ex plus ex minus .2ex}{1.5ex plus .2ex}





\fancyhead[RE, LO]{\nouppercase{\emph\leftmark}}
%\fancyfoot[C]{Departamento de Estatística}

%%%%%%%%%%%%%%%%%%%%%%%%%%%%%%%%%%%%%%%%%%%%%%%%%%%%%%%%%%%%%%%%%%%%%%%%%%%%%%%%%%%%%%%%%%%%%%%%%%%%%%%%%%
%%%%%%%%%%%%%%%%%%%%%%%%%%%%%%%%		Dedicatória			%%%%%%%%%%%%%%%%%%%%%%%%%%%%%%%%%%%%%%%%%%%%%%
%%%%%%%%%%%%%%%%%%%%%%%%%%%%%%%%%%%%%%%%%%%%%%%%%%%%%%%%%%%%%%%%%%%%%%%%%%%%%%%%%%%%%%%%%%%%%%%%%%%%%%%%%%

%``exemplo"

%\vspace*{13cm}

%\hfill{\begin{minipage}{10.4cm}
%incluir dedicatória (opcional)

%\end{minipage}}
%\newpage

%%%%%%%%%%%%%%%%%%%%%%%%%%%%%%%%%%%%%%%%%%%%%%%%%%%%%%%%%%%%%%%%%%%%%%%%%%%%%%%%%%%%%%%%%%%%%%%%%%%%%%%%%%
%%%%%%%%%%%%%%%%%%%%%%%%%%%%%%%%		Agradecimentos			%%%%%%%%%%%%%%%%%%%%%%%%%%%%%%%%%%%%%%%%%%
%%%%%%%%%%%%%%%%%%%%%%%%%%%%%%%%%%%%%%%%%%%%%%%%%%%%%%%%%%%%%%%%%%%%%%%%%%%%%%%%%%%%%%%%%%%%%%%%%%%%%%%%%%


\vspace*{2.5cm}

\begin{center}
 {\Huge \bfseries Agradecimentos}
\end{center}
\baselineskip 19.5pt 
\vspace*{1.5cm}

% bahia, gabriela, tassinari, Leandro, meu pai, unb, educação pública, Enap, os funcionários de serviços auxiliares da unb em especial o RU e a mim mesmo, embrapa à Embrapa Arroz e Feijão e em especial ao Dr. Flávio Breseghello  por ceder o banco de dados para nossas análises.

\newpage

%%%%%%%%%%%%%%%%%%%%%%%%%%%%%%%%%%%%%%%%%%%%%%%%%%%%%%%%%%%%%%%%%%%%%%%%%%%%%%%%%%%%%%%%%%%%%%%%%%%%%%%%%%
%%%%%%%%%%%%%%%%%%%%%%%%%%%%%%%%%%%%		Resumo			%%%%%%%%%%%%%%%%%%%%%%%%%%%%%%%%%%%%%%%%%%%%%%
%%%%%%%%%%%%%%%%%%%%%%%%%%%%%%%%%%%%%%%%%%%%%%%%%%%%%%%%%%%%%%%%%%%%%%%%%%%%%%%%%%%%%%%%%%%%%%%%%%%%%%%%%%

\vspace*{2.5cm}
\begin{center}
 {\Huge \bfseries Resumo}
\end{center}
\baselineskip 19.5pt 
\vspace*{1.5cm}

asdf

\vspace*{1.5cm}

Palavras-chave: asdf

  
\newpage


%%%%%%%%%%%%%%%%%%%%%%%%%%%%%%%%%%%%%%%%%%%%%%%%%%%%%%%%%%%%%%%%%%%%%%%%%%%%%%%%%%%%%%%%%%%%%%%%%%%%%%%%%%
%%%%%%%%%%%%%%%%%%%%%%%%%%%%%%%%		Tabelas/Figuras			%%%%%%%%%%%%%%%%%%%%%%%%%%%%%%%%%%%%%%%%%%
%%%%%%%%%%%%%%%%%%%%%%%%%%%%%%%%%%%%%%%%%%%%%%%%%%%%%%%%%%%%%%%%%%%%%%%%%%%%%%%%%%%%%%%%%%%%%%%%%%%%%%%%%%

\listoftables

\newpage

\listoffigures

\newpage

%Sumário
\tableofcontents

\newpage

\pagenumbering{arabic}
\setcounter{page}{8}

\chapter{Introdução}

\section{Motivação}

A domesticação de espécies silvestres de plantas para a agricultura é uma prática antiga e passou por diversas revoluções até os dias atuais, em que a genética biométrica e o melhoramento de precisão protagonizam a criação de cultivares e seleção de características de interesse \cite{melhora_precisa1}. Além disso, pressões como crescimento populacional \citep{hickey2019breeding}, redução de recursos naturais disponíveis, aquecimento global e uma variedade de consequências desses fatores \citep{jorasch2019} aumentam a necessidade de se produzir alimentos e outros recursos vegetais de forma incrementalmente eficiente. Uma das soluções para isso é justamente o melhoramento de precisão.

Neste contexto, o desenvolvimento e seleção de cultivares é associado a identificação de grupos ambientais (\textit{Target Population of Environments} ou TPE), permitindo que se aproveite ao máximo a característica de interesse \cite{chenu2015characterizing}. De fato, em posse da informação de que o ambiente em que a planta se desenvolve interfere em seu fenótipo (a característica de interesse, que é uma expressão gênica), cabe estudar a interação genótipos $\times$ ambientes ($G \times E$).

O estudo desse tipo de relação é potencializado com o uso de técnicas de Sistemas de Informações Geográficas -- SIG, como sensoriamento remoto, entre outros \cite{melhora_precisa1}. A disponibilização pública de dados coletados via satélite com diversos graus de granularidade permite a inclusão de mais covariáveis ambientais como área cultivada, cobertura vegetal, temperatura, entre outros dados geofísicos\footnote{Por exemplo, o serviço Google Earth Engine disponibiliza seu catálogo em https://developers.google.com/earth-engine/datasets/}.

A proposta de \citep{resende2020enviromics}, que será usada de estudo de caso, é expandir o uso de TPE para um estudo ômico do ambiente, daí \textit{ambientômica}. Os autores propõem o uso de modelos hierárquicos, e o conceito de ambientipagem, resultante de agrupamentos ambientais, para predição de performance de genótipos não observados. Isto permite, por exemplo, recomendar o melhor genótipo de um determinado cultivar para uma região em que jamais foi cultivado e assim tornar a região produtiva.

\section{Objetivos}

O objetivo geral deste trabalho de conclusão de curso é estudar o uso de modelos lineares hierárquicos (ou mistos) para recomendação de genótipos de um determinado cultivar em uma região delimitada e ambientipada, isto é, com dados sobre a maior quantidade de características ambientais possível. Pretende-se revisar metodologicamente o estudo de \citep{resende2020enviromics}, detalhando o processo de modelagem e sua adequação, bem como comparar computacionalmente variações do modelo utilizado.

 
Os objetivos específicos são:

  \begin{itemize}
  \item Explorar a técnica de modelagem estatística via modelos lineares hierárquicos incluindo efeitos aleatórios;
  \item Explorar os conceitos necessários para aplicação do modelo ao contexto de melhoramento de plantas e ambientômica;
  \item Comparar a adequação do modelo original dos autores com um modelo que faça composição de marcadores ambientômicos utilizando técnicas de redução de dimensionalidade.
 \end{itemize}
 
 
 % REFERENCIAL TEÓRICO
 %%%%%%%%%%%%%%%%%%%%%%%%%%%%%%%%%%%%%%%%%%%%%%%%%%%%%%%%%%%%%%%%%%%%%%%%%%%%%%%%
 
 \chapter{Referencial Teórico}
 
 \section{Análise de Agrupamentos}
 
 \section{Redução de dimensionalidade}
 
 De acordo com \cite{morettindatasci}, técnicas de redução de dimensionalidade são utilizadas quando há um grande número de covariáveis e se deseja estudar as unidades observadas com base em sua estrutura de dependência multivariada. Dentre as técnicas disponíveis, foi escolhida a Análise de Componentes Principais (\textit{PCA} --- \textit{Principal Component Analysis}), que busca expressar as covariáveis originais em termos de outras variáveis não correlacionadas e que resumam as informações contidas em um conjunto de variáveis. Em outras palavras, ``as $p$ covariáveis originais $(X_1, \dots , X_p)$ são transformadas em $p$ componentes princiais não correlacionadas $(Y_1, \dots, Y_p)$ de modo que $Y_1$ é aquela que explica a maior parcela da variabilidade total dos dados originais, $Y_2$ explica a segunda maior parcila e assim por diante'' \citep{metodosmultivariados_artes}. 
 
Considerando $\boldsymbol{x}$ o vertor de covariáveis originais $\boldsymbol{x}^\top = (X_1, \dots, X_p)$ e $\text{Cov}(x) = \boldsymbol{\Sigma}$, as componentes principais são obtidas pela decomposição espectral da matriz de covariâncias: $\boldsymbol{\Sigma} = \boldsymbol{\Gamma \Lambda \Gamma}^\top$. Dessa forma $\boldsymbol{\Lambda}$ é uma matriz diagonal dos autovalores e $\boldsymbol{\Gamma}$ é uma matriz ortogonal (portanto $\boldsymbol{\Gamma \Gamma}^\top = \boldsymbol{I}$) dos autovetores de $\boldsymbol{\Sigma}$. Obtém-se daí os pares de autovalores e autovetores ortogonais normalizados $(\lambda_i, \boldsymbol{\alpha}_i), i = 1, \dots, p,$, ordenados de modo que $\lambda_1 \geq \dots \geq \lambda_p \geq 0$. Dessa forma, pode-se expressar a $i$-ésima componente principal por

\begin{equation}
	Y_i = \boldsymbol{\alpha}_i^\top \boldsymbol{x} = \alpha_{i1}X_1 + \dots + \alpha_{ip}X_p, \quad i = 1, \dots , p.
\end{equation}

Caso as covariáveis tenham variância ou valores em escalas muito discrepantes, é possível normalizá-las e obter as componentes principais a partir da matriz de correlações $\rho$ e seus pares de autovalores e autovetores $(\gamma_i, \boldsymbol{\epsilon}_i)$. As covariáveis são padronizadas utilizando

\begin{align}
	\boldsymbol{z} = \left( \boldsymbol{V}^{1/2}\right)^{-1} (\boldsymbol{x} - \boldsymbol{\mu}),
\end{align}

\noindent em que $\boldsymbol{V}^{1/2} = \text{diag}(\sigma_1, \dots, \sigma_p)$ e $\boldsymbol{\mu}$ é o vetor de médias. Agora tem-se $\text{Cov}(\boldsymbol{z}) = \rho$, a matriz de correlações para as covariáveis padronizadas. Finalmente, as componentes são dadas por

\begin{align}
	Y_i = \boldsymbol{\epsilon}_i^\top \left( \boldsymbol{V}^{1/2}\right)^{-1} (\boldsymbol{x} - \boldsymbol{\mu}).
\end{align}

A obtenção de componentes principais para o sistema garante algumas propriedades interessantes como

\begin{align}
	\sum\limits_{j = 1}^{p} \text{Var}\left( X_j \right) = \text{tr} \left( \boldsymbol{\Sigma} \right)  = \text{tr} \left( \boldsymbol{\Gamma \Lambda \Gamma^\top} \right) = \text{tr} \left( \boldsymbol{\Lambda} \right) =  \sum\limits_{i = 1}^{p} \text{Var}\left( Y_i \right).
\end{align}

\noindent Ou seja, a variância total do sistema é mantida após obtenção das componentes principais, o que também vale quando se utiliza a matriz de correlações.   

Efetivamente o que será utilizado serão os estimadores amostrais para as as componentes principais, para as matrizes de covariância e correlação e seus pares de autovalores e autovetores; $\hat{Y}_i$, $\boldsymbol{S}$, $\boldsymbol{R}$ e $(\hat{\gamma}_i, \hat{\boldsymbol{\epsilon}}_i)$ respectivamente.

Quando o intuito da obtenção das componentes principais é reduzir a dimensionalidade do sistema e assim manter uma quantidade menor de variáveis que explique uma parcela razoável da variância do sistema, deve-se adotar um critério de determinação do número de componentes principais a serem retidas. \cite{metodosmultivariados_artes} apresentam:

\begin{itemize}
	\item Critério de Kaiser, segundo a qual se mantém as componentes com autovalores superiores a 1 --- ou seja, componentes que explicariam mais variância do que uma covariável original individual; \\
	\item Reter componentes que acumulem ao menos uma certa percentagem de variância explicada do total; \\
	\item Reter componentes que acumulem ao menos uma certa percentagem da variância explicada de cada variável original.
\end{itemize}

\cite{morettindatasci} apresentam ainda a seleção baseada no fator de aceleração do teste do cotovelo (utilização do \textit{screeplot}). Considerando $af(i) = f''(i) = f(i+1) - 2f(i) - f(i-1)$, $i = 1, \dots, p-1$ o fator de aceleração, o número de fatores retidos corresponde à posição anterior em que $af(i)$ é máximo.



\section{Modelos Lineares}

Modelos Lineares apresentam uma relação estocástica entre duas ou mais variáveis. Sua forma simples com efeitos fixos pode ser representada da forma

PASSAR A NOTAÇÃO PARA MATRICIAL

\begin{align}
	\boldsymbol{Y} = \boldsymbol{X \beta + \varepsilon} \label{relacao_linear_basica}
\end{align}

\noindent em que $\boldsymbol{Y}$ é o vetor da variável resposta, $\boldsymbol{\beta}$ é o vetor de coeficientes associados às covariáveis $\boldsymbol{X} = \left( \boldsymbol{X}_1, \dots, \boldsymbol{X}_p \right)^\top$ e $\boldsymbol{\varepsilon}$ é o vetor de erros estocásticos associado às observações. Ao final do processo de modelagem obtém-se um modelo da forma


\begin{align}
	\hat{\boldsymbol{Y}} = \boldsymbol{X \hat{\beta}}, \label{modelo_linear_basico}
\end{align}

\noindent em que $\hat{\boldsymbol{\beta}}$ são estimadores obtidos pelo método de mínimos quadrados ordinários ou máxima verossimilhança para $\boldsymbol{\beta}$ \citep{kutner2005applied}.

Este modelo de regressão, que também pode ser chamado de \textit{modelo de efeitos fixos}, exige uma série de suposições a respeito da componente aleatória que são avaliadas na etapa diagnóstica da modelagem como heteroscedasticidade, independência e distribuição Normal. Caso uma ou mais suposições não possam ser verificadas, se observe colinearidade entre as covariáveis do modelo, pontos de alavancagem ruins, ou outros comprometimentos do modelo, recorre-se a  transformações, redução de dimensionalidade, entre outros. Essas medidas, assim como variáveis que indiquem grupos aos quais as observações pertencem, dados categóricos, entre outros, comumente aumentam a complexidade da interpretação do modelo e exigem tratamentos inferenciais diferentes \citep{hox2017multilevel}.


\subsection{Modelos Lineares Hierárquicos}

REVISAR TODA ESSA SEÇÃO COM DEMIDENKO + CADERNOS UFV

Frequentemente pesquisas em domínios variados do conhecimento estudam fenômenos em que as unidades de análise são agregadas em categorias \citep{adewale2007understanding, mcmahon2007scales}. Em alguns casos, as unidades são aninhadas em um ou mais níveis de categorias. Esses diferentes níveis de análise, indivíduos ou grupos, e suas características ou intervenções sobre níveis diferentes requerem diferentes formas de representação e técnicas de inferência que comportem adequadamente as estruturas de covariância envolvidas.

Modelos multinível substituem duas práticas comuns na utilização de regressões lineares: transformação de variáveis categóricas em variáveis binárias (*dummy*) e planificação do nível de análise, ou seja, utilização de medidas de grupos e indivíduos como descritores diretos da unidade de análise. A utilização de um modelo multinível permite a construção de estimadores que contornam essas estratégias e representam melhor indivíduos e grupos no contexto de suas características \citep{hox2017multilevel, gelman_hill_2007}. Esse tipo de modelo linear pode ser representado na forma


\begin{align}
	y_{ij} = \beta_{0j} + \sum\limits_{k = 1}^{p} \beta_{kj} \, x_{ik} + \varepsilon_{ij}, \label{213}
\end{align}
%
%\noindent em que $y_{ij}$ é a variável resposta a nível de indivíduo, $\beta_{0j}$ é o intercepto para o grupo $j = 1, 2, \dots, J$ a que esse indivíduo pertence, $\beta_{kj}$ são os coeficientes para cada covariável de nível individual $x_{ik}, k = 0, 1, \dots, K$ e $\varepsilon_{ij} \overset{iid}{\sim} N(0, \sigma^2_\varepsilon)$ é a componente aleatória para indivíduos.
%
%No entanto, $\boldsymbol{\beta}$ é estimado a partir das covariáveis de nível superior. Se considerarmos apenas um nível, cada $\beta_{kj}$ é expresso por
%
%
%\begin{align}
%	\beta_{kj} & = \gamma_{k0} + \sum\limits_{l = 1}^{L} \gamma_{klj} \, z_{lj} + u_{kj}, \quad l = 1, 2, \dots, L \label{214}
%\end{align}
%
%
%\noindent em que $\gamma_{k0}$ é o componente fixo, $\gamma_{klj}$ é o coeficiente para cada covariável $z_{l(.)}$ de nível superior $j$ e $u_{kj} \overset{iid}{\sim} N(0, \sigma^2_{u_{kj}})$ é a componente aleatória de cada $\beta_{k(.)}$ do grupo $j$. Uma propriedade deste tipo de modelo é que $E(\gamma_{klj}) = 0$, de modo que é possível depreender da equação (\ref{214}) que $\beta_{k} \sim N(0, \sigma^2_{u_{k}})$.
%
%Se for considerado o caso simplificado de apenas uma covariável de cada nível e substituirmos (\ref{214}) em (\label{213}), obtém-se
%
%
%\begin{align}
%	y_{ij} = \gamma_{00} + \gamma_{01}z_{1j} + \gamma_{10}x_{1ij} + \gamma_{11}z_{1j}x_{1i} + u_{1j}x_{1ij} + \varepsilon_{ij} + u_{0j}. \label{215}
%\end{align}
%
%É imediato da equação (\ref{215}) que:
%
%
%\begin{itemize}
%	\item Existe um intercepto geral -- $\gamma_{00}$;
%	\item Existem efeitos que agem exclusivamente sobre variáveis de um nível hierárquico específico -- $\gamma_{01}z_{1j}$ e $\gamma_{10}x_{1ij}$;
%	\item Existe um efeito de mediação do comportamento do grupo sobre a unidade de observação -- $\gamma_{11}z_{1j}x_{1i}$;
%	\item Existe uma componente de variância do grupo que incide sobre o comportamento da unidade -- $u_{1j}x_{1ij}$; e
%	\item Existem componentes de variância entre unidades e entre grupos -- $\varepsilon_{ij}$ e $u_{0j}$ respectivamente.
%\end{itemize}
%
%As componentes de variância deste modelo são obtidas a partir do modelo ajustado apenas com os interceptos \cite{hox2017multilevel} de $\varepsilon_{ij}$ e $u_{0j}$, de modo que se pode calcular a proporção de variância no segundo nível da hierarquia, entre agrupamentos. Essa estatística pode ser interpretada como uma correlação entre indivíduos de um mesmo grupo, presumidamente mais similares entre si quando comparados a outro grupo. Essa medida é chamada de correlação intraclasse e, para o caso de apenas dois níveis, é dada por
%
%
%\begin{align}
%	\rho = \frac{\sigma^2_{u_0}}{\sigma^2_{u_0}+\sigma^2_{\varepsilon}}. \label{224}
%\end{align}




\chapter{Metodologia}


COLOCAR CLUSTER REGIONAL NO MODELO HIERARQUICO



\section{Conjuntos de dados}

%1995-2021
%trial, syst, year, st, location = loc, type, desing
%n rep, mean, rep, gen, gy
% dissertação do bahia sobre o que entra como efeito aleatório
% inserir cluster como agrupamento

O conjunto de dados experimentais utilizado foi o \textit{Embrapa Rice Breeding Dataset} (ERBD), desenvolvido e cedido pela Empresa Brasileira de Pesquisa Agropecuária (Embrapa) para pesquisas com cultivares de arroz (\textit{Oryza sativa L.}). A base de dados cedida compreende coleta de dados desde 1982, mas os dados utilizados neste trabalho compreendem apenas os anos de 1995 a 2021 por recomendação dos pesquisadores da Empresa. A base é extensamente documentada em \cite{breseghello2011} e \cite{breseghello2021} e os dicionários desses dados estão disponíveis no Apêndice, tabelas \ref{tab:dicionarioerbd1} e \ref{tab:dicionarioerbd2}.

Para o conjunto de dados ambientais foram coletados dados de 393 covariáveis, das quais 19 são oriundas do repositório WorldClim, 130 do NasaPower e 244 do SoilGrids. Todos os dados ambientais advém de medições via satélite e foram obtidos mediante consulta às API (\textit{Application Programming Interface}) disponibilizadas pelas organizações. O dicionário de dados contendo a fonte e definição de todas as 393 covariaveis está disponível no Apêndice, tabela \ref{tab:dicionarioambientais}.  As três fontes são descritas a seguir:

A \textit{SoilGrids} \footnote{https://www.isric.org/explore/soilgrids} é um projeto da \textit{International Soil Reference and Information Centre} é um sistema para mapeamento digital do solo e faz predições  para as suas distribuições de forma global. Suas variáveis são dadas em termos de distribuições para seis horizontes de profundidade no solo e alguns de seus quantis.

\textit{NASAPOWER} (\textit{Prediction Of Worldwide Energy Resources})\footnote{https://power.larc.nasa.gov/} é um projeto que foi desenvolvido para melhorar ou criar dados de sistemas de satélites e tem três comunidades-alov: energia renovável, construções sustentáveis e agroclimatologia. Os dados utilizados são aqueles destinados ao último público.

Finalmente, \textit{WorldClim}\footnote{https://www.worldclim.org/data/index.html} é uma base de dados com dados de tempo e clima de alta resolução. As variáveis são derivadas de de valores mensais de precipitação e temperatura, seja uma tendência anual ou de fatores ambientais limitantes ou extremos (em termos de temperatura, por exemplo).

Os dados de satélite dessas três fontes foram coletados em 87.155 pontos em todo o país, mas não correspondem exatamente aos 149 pontos de experimento. Embora existam métodos apropriados para estimar toda a superfície do país e, em seguida, utilizar a estimação para os pontos de experimento \citep{cokriging}, este não é o foco do estudo, nem se pretende seguir essa metodologia específica. Consequentemente, o erro de estimação das covariáveis para os pontos de experimento será ignorado. Para a interpolação das covariáveis para os pontos de experimento, optou-se por calcular a média das covariáveis dos três pontos mais próximos.


\section{Software}

Para todas as análises e construção de gráficos foi utilizada a linguagem R VERSAO no ambiente de desenvolvimento RStudio VERSAO e os seguintes pacotes:

\begin{itemize}
	\item pacote1 versao
	\item pacote2 versao
\end{itemize}


\section{Análise Exploratória}

Não há dados de parentalidade, então não conseguimos usar dados de pedigree. a variável de agrupamento utilizada será o próprio genótipo

Podemos analisar:

Ambiente
Genótipos
Marcadores ambientômicos

\section{Análise de Agrupamentos}

MONTAR CLUSTERS DE REGIÕES DO BRASIL DE ACORDO COM AS VARIÁVEIS AMBIENTAIS

\subsection{Redução de dimensionalidade}

Descrição da redução de dimensionalidade

\section{Modelagem}

\subsection{Modelos Mistos}

Motivo do uso de modelos mistos
Descrição do que é o modelo linear misto, suas características, método de estimação, forma de aproximação numérica no R,


\subsection{Montagem dos Marcadores Ambientômicos}

MONTAR OS INDICADORES AMBIENTOMICOS, MAS INCLUINDO TAMBÉMA CLUSTERIZAÇÃO DAS REGIÕES DO BRASIL DE ACORDO COM AS VARIÁVEIS AMBIENTAIS

falar das simulacoes

O modelo do Tassinari e a probabilidade de selecionar cada uma das covariáveis para o modelo.


% RESULTADOS
%%%%%%%%%%%%%%%%%%%%%%%%%%%%%%%%%%%%%%%%%%%%%%%%%%%%%%%%%%%%%%%%%%%%%%%%%%%

\chapter{Resultados}

análise exploratória, ambiente, genótipos, marcadores ambientômicos, modelo e desempenho

\section{Análise Exploratória}

\section{Agrupamentos}

\section{Redução de Dimensionalidade}

\section{Composição dos Marcadores Ambientômicos}

\section{Modelo Misto}

\chapter{Conclusões}



\newpage

% REFERENCIAS
%%%%%%%%%%%%%%%%%%%%%%%%%%%%%%%%%%%%%%%%%%%%%

% AJUSTAR REFERENCIAS PARA MODELO ABNT

\addcontentsline{toc}{chapter}{\textbf{Referências}}
\bibliographystyle{bbs2}
\bibliography{biblio}

% APENDICE
%%%%%%%%%%%%%%%%%%%%%%%%%%%%%%%%%%%%%%%%%%%%%
\chapter*{\textbf{Apêndice}}


	\begin{longtable}{@{} p{4cm} p{4cm} p{8cm} @{}} 
		\caption{Dicionário de variáveis do ERBD relacionadas aos ensaios}
		\label{tab:dicionarioerbd1} \\
		\toprule
		\textbf{Variável} & \textbf{Nome}                    & \textbf{Detalhes}                                                           \\* \midrule
		\endfirsthead
		%
		\endhead
		%
		\bottomrule
		\endfoot
		%
		\endlastfoot
		%
		TRIAL                & Código do ensaio        & String único que identifica o ensaio                               \\
		SYST   & Sistema de cultivo           & Indica tanto o subprograma de melhoramento quanto o ambiente do ensaio. Níveis: Irrigado ou de Sequeiro            \\
		YEAR                 & Ano do ensaio           & Ano de preparação do ensaio. Ex: 2005: temporada 2005/2006         \\
		DATE                 & Data de plantio         & Dia de plantio de sementes secas. Formato DD/MM/AAAA               \\
		ST                   & Estado do Brasil        & Estado do Brasil onde o ensaio foi conduzido                       \\
		LOCATION             & Local de plantio        & Nome do município onde o ensaio foi conduzido                      \\
		LOC                  & Local de plantio        & Termo abreviado que indica o município                             \\
		TYPE   & Tipo de ensaio               & Tipo de ensaio. ER: Ensaios Regionais de Rendimento; VCU: Valor de Cultivo e Uso (Ensaios Avançados de Rendimento) \\
		DESIGN & Desenho Experimental         & O desenho estatístico do ensaio. RCB: delineamento de blocos completos ao acaso; LAT: delineamento em látice       \\
		MEAN   & Média de rendimento de grãos & Média geral do ensaio do rendimento de grãos (kg ha$^{-1}$)                                       \\
		H$^2$ & Hereditariedade         & Hereditariedade de sentido amplo do rendimento de grãos            \\
		CV                   & Coeficiente de Variação & Coeficiente de variação experimental para rendimento de grãos (\%) \\ 
		\bottomrule
		\multicolumn{3}{@{}l}{\footnotesize\textit{Fonte: \cite{breseghello2021}}}\\
	\end{longtable}

%%%%%%%%%%%%%%%%%%%%%%%%%%%%%%%%%%%%
%%%     		TABELA 2
%%%%%%%%%%%%%%%%%%%%%%%%%%%%%%%%%%%%

\begin{longtable}{@{} p{2.5cm} p{2.5cm} p{2.5cm} p{8.5cm} @{}} 
	\caption{Dicionário de variáveis do ERBD das unidades experimentais}
	\label{tab:dicionarioerbd2} \\
	\toprule
	\textbf{Nome na base} & \textbf{Nome da variável} & \textbf{Tipo}       & \textbf{Detalhes}                                                                         \\ \midrule
	\endfirsthead
	%
	\endhead
	%
	\bottomrule
	\endfoot
	%
	\endlastfoot
	%
	TRIAL  & Código do ensaio          & Link para Metadados & String único que identifica o ensaio                                                      \\
	REP    & Número da Repetição       & Fator de Design     & Inteiro que indica a repetição dentro do ensaio                                           \\
	BLO    & Número do Bloco           & Fator de Design     & Inteiro que indica o bloco dentro da repetição (apenas em delineamento em látice)         \\
	GEN    & Nome do Genótipo          & Fator Experimental  & Identificação do germoplasma (linhagem pura, variedade local ou cultivar)                 \\
	GY     & Rendimento de grãos       & Numérico            & Peso do arroz em casca com 13\% de umidade, em kg ha$^{-1}$                              \\
	PHT    & Altura da Planta          & Numérico            & Altura da planta do solo até a ponta da panícula primária, no estágio pré-colheita, em cm \\
	DTF    & Dias até a Floração       & Numérico            & Número de dias desde o plantio de sementes secas até 50\% das plantas estarem floridas    \\
	LOD    & Tombamento                & Escores de 1 a 9\footnote{Escores mais altos indicam níveis crescentes de tombamento ou doença.}    & Nível de tombamento da copa da parcela, avaliado no estágio pré-colheita                  \\
	LBL    & Blast na Folha            & Escores de 1 a 9    & Severidade da doença de brusone do arroz, avaliada em folhas no estágio vegetativo       \\
	PBL    & Blast na Panícula         & Escores de 1 a 9    & Severidade da doença de brusone do arroz, avaliada em panículas no estágio pré-colheita  \\
	BSP    & Mancha Parda              & Escores de 1 a 9    & Severidade da doença causada por \textit{Bipolaris oryzae}, avaliada em folhas no estágio pré-colheita                            \\
	LSC    & Escaldadura da Folha      & Escores de 1 a 9    & Severidade da doença causada por \textit{Monographella albescens}, avaliada em folhas no estágio pré-colheita                     \\
	GDS    & Descoloração do Grão      & Escores de 1 a 9    & Severidade do escurecimento ou manchas nos grãos, causada por vários fungos, avaliada nas glumas no estágio pré-colheita \\
	\bottomrule
	\multicolumn{4}{@{}l}{\footnotesize\textit{Fonte: \cite{breseghello2021}}}\\
\end{longtable}

%%%%%%%%%%%%%%%%%%%%%%%%%%%%%%%%%%%%
%%%     		TABELA 3
%%%%%%%%%%%%%%%%%%%%%%%%%%%%%%%%%%%%

\begin{longtable}{@{} p{4cm} p{4cm} p{8cm} @{}} 
	\caption{Dicionário de variáveis ambientais}
	\label{tab:dicionarioambientais} \\
	\toprule
	\textbf{Nome na base} & \textbf{Nome da variável} & \textbf{Descrição} \\ \midrule
	\endfirsthead
	%
	\endhead
	%
	\bottomrule
	\endfoot
	%
	\endlastfoot
	%
	WorldClim &
	bio\_1 &
	Temperatura Média Anual \\
	WorldClim &
	bio\_2 &
	Amplitude Diurna Média (Média das diferenças mensais (temp máx - temp mín)) \\
	WorldClim &
	bio\_3 &
	Isotermia (BIO2/BIO7) (×100) \\
	WorldClim &
	bio\_4 &
	Sazonalidade da Temperatura (desvio padrão ×100) \\
	WorldClim &
	bio\_5 &
	Temperatura Máxima do Mês Mais Quente \\
	WorldClim &
	bio\_6 &
	Temperatura Mínima do Mês Mais Frio \\
	WorldClim &
	bio\_7 &
	Amplitude Térmica Anual (BIO5-BIO6) \\
	WorldClim &
	bio\_8 &
	Temperatura Média do Trimestre Mais Chuvoso \\
	WorldClim &
	bio\_9 &
	Temperatura Média do Trimestre Mais Seco \\
	WorldClim &
	bio\_10 &
	Temperatura Média do Trimestre Mais Quente \\
	WorldClim &
	bio\_11 &
	Temperatura Média do Trimestre Mais Frio \\
	WorldClim &
	bio\_12 &
	Precipitação Anual \\
	WorldClim &
	bio\_13 &
	Precipitação do Mês Mais Chuvoso \\
	WorldClim &
	bio\_14 &
	Precipitação do Mês Mais Seco \\
	WorldClim &
	bio\_15 &
	Sazonalidade da Precipitação (Coeficiente de Variação) \\
	WorldClim &
	bio\_16 &
	Precipitação do Trimestre Mais Chuvoso \\
	WorldClim &
	bio\_17 &
	Precipitação do Trimestre Mais Seco \\
	WorldClim &
	bio\_18 &
	Precipitação do Trimestre Mais Quente \\
	WorldClim &
	bio\_19 &
	Precipitação do Trimestre Mais Frio \\
	SoilGrids &
	bdod\_0-5cm\_mean &
	Densidade aparente da fração fina do solo em cg/cm³ para centímetros 0 a 5 da superfície – média \\
	SoilGrids &
	bdod\_0-5cm\_Q0p05 &
	Densidade aparente da fração fina do solo em cg/cm³ para centímetros 0 a 5 da superfície – predição para quantil 0,05 \\
	SoilGrids &
	bdod\_0-5cm\_Q0p5 &
	Densidade aparente da fração fina do solo em cg/cm³ para centímetros 0 a 5 da superfície – predição para quantil 0,5 \\
	SoilGrids &
	bdod\_0-5cm\_Q0p95 &
	Densidade aparente da fração fina do solo em cg/cm³ para centímetros 0 a 5 da superfície – predição para quantil 0,95 \\
	SoilGrids &
	bdod\_5-15cm\_mean &
	Densidade aparente da fração fina do solo em cg/cm³ para centímetros 5 a 15 da superfície – média \\
	SoilGrids &
	bdod\_5-15cm\_Q0p05 &
	Densidade aparente da fração fina do solo em cg/cm³ para centímetros 5 a 15 da superfície – predição para quantil 0,05 \\
	SoilGrids &
	bdod\_5-15cm\_Q0p5 &
	Densidade aparente da fração fina do solo em cg/cm³ para centímetros 5 a 15 da superfície – predição para quantil 0,5 \\
	SoilGrids &
	bdod\_5-15cm\_Q0p95 &
	Densidade aparente da fração fina do solo em cg/cm³ para centímetros 5 a 15 da superfície – predição para quantil 0,95 \\
	SoilGrids &
	bdod\_15-30cm\_mean &
	Densidade aparente da fração fina do solo em cg/cm³ para centímetros 15 a 30 da superfície – média \\
	SoilGrids &
	bdod\_15-30cm\_Q0p05 &
	Densidade aparente da fração fina do solo em cg/cm³ para centímetros 15 a 30 da superfície – predição para quantil 0,05 \\
	SoilGrids &
	bdod\_15-30cm\_Q0p5 &
	Densidade aparente da fração fina do solo em cg/cm³ para centímetros 15 a 30 da superfície – predição para quantil 0,5 \\
	SoilGrids &
	bdod\_15-30cm\_Q0p95 &
	Densidade aparente da fração fina do solo em cg/cm³ para centímetros 15 a 30 da superfície – predição para quantil 0,95 \\
	SoilGrids &
	bdod\_30-60cm\_mean &
	Densidade aparente da fração fina do solo em cg/cm³ para centímetros 15 a 30 da superfície – média \\
	SoilGrids &
	bdod\_30-60cm\_Q0p05 &
	Densidade aparente da fração fina do solo em cg/cm³ para centímetros 30 a 60 da superfície – predição para quantil 0,05 \\
	SoilGrids &
	bdod\_30-60cm\_Q0p5 &
	Densidade aparente da fração fina do solo em cg/cm³ para centímetros 30 a 60 da superfície – predição para quantil 0,5 \\
	SoilGrids &
	bdod\_30-60cm\_Q0p95 &
	Densidade aparente da fração fina do solo em cg/cm³ para centímetros 30 a 60 da superfície – predição para quantil 0,95 \\
	SoilGrids &
	bdod\_60-100cm\_mean &
	Densidade aparente da fração fina do solo em cg/cm³ para centímetros 60 a 100 da superfície – média \\
	SoilGrids &
	bdod\_60-100cm\_Q0p05 &
	Densidade aparente da fração fina do solo em cg/cm³ para centímetros 60 a 100 da superfície – predição para quantil 0,05 \\
	SoilGrids &
	bdod\_60-100cm\_Q0p5 &
	Densidade aparente da fração fina do solo em cg/cm³ para centímetros 60 a 100 da superfície – predição para quantil 0,5 \\
	SoilGrids &
	bdod\_60-100cm\_Q0p95 &
	Densidade aparente da fração fina do solo em cg/cm³ para centímetros 60 a 100 da superfície – predição para quantil 0,95 \\
	SoilGrids &
	bdod\_100-200cm\_mean &
	Densidade aparente da fração fina do solo em cg/cm³ para centímetros 100 a 200 da superfície – média \\
	SoilGrids &
	bdod\_100-200cm\_Q0p05 &
	Densidade aparente da fração fina do solo em cg/cm³ para centímetros 100 a 200 da superfície – predição para quantil 0,05 \\
	SoilGrids &
	bdod\_100-200cm\_Q0p5 &
	Densidade aparente da fração fina do solo em cg/cm³ para centímetros 100 a 200 da superfície – predição para quantil 0,5 \\
	SoilGrids &
	bdod\_100-200cm\_Q0p95 &
	Densidade aparente da fração fina do solo em cg/cm³ para centímetros 100 a 200 da superfície – predição para quantil 0,95 \\
	SoilGrids &
	cec\_0-5cm\_mean &
	Capacidade de Troca de Cátions do solo em mmol(c)/kg para centímetros 0 a 5 da superfície – média \\
	SoilGrids &
	cec\_0-5cm\_Q0p05 &
	Capacidade de Troca de Cátions do solo em mmol(c)/kg para centímetros 0 a 5 da superfície – predição para quantil 0,05 \\
	SoilGrids &
	cec\_0-5cm\_Q0p5 &
	Capacidade de Troca de Cátions do solo em mmol(c)/kg para centímetros 0 a 5 da superfície – predição para quantil 0,5 \\
	SoilGrids &
	cec\_0-5cm\_Q0p95 &
	Capacidade de Troca de Cátions do solo em mmol(c)/kg para centímetros 0 a 5 da superfície – predição para quantil 0,95 \\
	SoilGrids &
	cec\_5-15cm\_mean &
	Capacidade de Troca de Cátions do solo em mmol(c)/kg para centímetros 5 a 15 da superfície – média \\
	SoilGrids &
	cec\_5-15cm\_Q0p05 &
	Capacidade de Troca de Cátions do solo em mmol(c)/kg para centímetros 5 a 15 da superfície – predição para quantil 0,05 \\
	SoilGrids &
	cec\_5-15cm\_Q0p5 &
	Capacidade de Troca de Cátions do solo em mmol(c)/kg para centímetros 5 a 15 da superfície – predição para quantil 0,5 \\
	SoilGrids &
	cec\_5-15cm\_Q0p95 &
	Capacidade de Troca de Cátions do solo em mmol(c)/kg para centímetros 5 a 15 da superfície – predição para quantil 0,95 \\
	SoilGrids &
	cec\_15-30cm\_mean &
	Capacidade de Troca de Cátions do solo em mmol(c)/kg para centímetros 15 a 30 da superfície – média \\
	SoilGrids &
	cec\_15-30cm\_Q0p05 &
	Capacidade de Troca de Cátions do solo em mmol(c)/kg para centímetros 15 a 30 da superfície – predição para quantil 0,05 \\
	SoilGrids &
	cec\_15-30cm\_Q0p5 &
	Capacidade de Troca de Cátions do solo em mmol(c)/kg para centímetros 15 a 30 da superfície – predição para quantil 0,5 \\
	SoilGrids &
	cec\_15-30cm\_Q0p95 &
	Capacidade de Troca de Cátions do solo em mmol(c)/kg para centímetros 15 a 30 da superfície – predição para quantil 0,95 \\
	SoilGrids &
	cec\_30-60cm\_mean &
	Capacidade de Troca de Cátions do solo em mmol(c)/kg para centímetros 15 a 30 da superfície – média \\
	SoilGrids &
	cec\_30-60cm\_Q0p05 &
	Capacidade de Troca de Cátions do solo em mmol(c)/kg para centímetros 30 a 60 da superfície – predição para quantil 0,05 \\
	SoilGrids &
	cec\_30-60cm\_Q0p5 &
	Capacidade de Troca de Cátions do solo em mmol(c)/kg para centímetros 30 a 60 da superfície – predição para quantil 0,5 \\
	SoilGrids &
	cec\_30-60cm\_Q0p95 &
	Capacidade de Troca de Cátions do solo em mmol(c)/kg para centímetros 30 a 60 da superfície – predição para quantil 0,95 \\
	SoilGrids &
	cec\_60-100cm\_mean &
	Capacidade de Troca de Cátions do solo em mmol(c)/kg para centímetros 60 a 100 da superfície – média \\
	SoilGrids &
	cec\_60-100cm\_Q0p05 &
	Capacidade de Troca de Cátions do solo em mmol(c)/kg para centímetros 60 a 100 da superfície – predição para quantil 0,05 \\
	SoilGrids &
	cec\_60-100cm\_Q0p5 &
	Capacidade de Troca de Cátions do solo em mmol(c)/kg para centímetros 60 a 100 da superfície – predição para quantil 0,5 \\
	SoilGrids &
	cec\_60-100cm\_Q0p95 &
	Capacidade de Troca de Cátions do solo em mmol(c)/kg para centímetros 60 a 100 da superfície – predição para quantil 0,95 \\
	SoilGrids &
	cec\_100-200cm\_mean &
	Capacidade de Troca de Cátions do solo em mmol(c)/kg para centímetros 100 a 200 da superfície – média \\
	SoilGrids &
	cec\_100-200cm\_Q0p05 &
	Capacidade de Troca de Cátions do solo em mmol(c)/kg para centímetros 100 a 200 da superfície – predição para quantil 0,05 \\
	SoilGrids &
	cec\_100-200cm\_Q0p5 &
	Capacidade de Troca de Cátions do solo em mmol(c)/kg para centímetros 100 a 200 da superfície – predição para quantil 0,5 \\
	SoilGrids &
	cec\_100-200cm\_Q0p95 &
	Capacidade de Troca de Cátions do solo em mmol(c)/kg para centímetros 100 a 200 da superfície – predição para quantil 0,95 \\
	SoilGrids &
	cfvo\_0-5cm\_mean &
	Fração volumétrica de fragmentos grosseiros (\textgreater 2 mm) em cm3/dm3 (vol‰) para centímetros 0 a 5 da superfície - média \\
	SoilGrids &
	cfvo\_0-5cm\_Q0p05 &
	Fração volumétrica de fragmentos grosseiros (\textgreater 2 mm) em cm3/dm3 (vol‰) para centímetros 0 a 5 da superfície – predição para quantil 0,05 \\
	SoilGrids &
	cfvo\_0-5cm\_Q0p5 &
	Fração volumétrica de fragmentos grosseiros (\textgreater 2 mm) em cm3/dm3 (vol‰) para centímetros 0 a 5 da superfície – predição para quantil 0,5 \\
	SoilGrids &
	cfvo\_0-5cm\_Q0p95 &
	Fração volumétrica de fragmentos grosseiros (\textgreater 2 mm) em cm3/dm3 (vol‰) para centímetros 0 a 5 da superfície – predição para quantil 0,95 \\
	SoilGrids &
	cfvo\_5-15cm\_mean &
	Fração volumétrica de fragmentos grosseiros (\textgreater 2 mm) em cm3/dm3 (vol‰) para centímetros 5 a 15 da superfície – média \\
	SoilGrids &
	cfvo\_5-15cm\_Q0p05 &
	Fração volumétrica de fragmentos grosseiros (\textgreater 2 mm) em cm3/dm3 (vol‰) para centímetros 5 a 15 da superfície – predição para quantil 0,05 \\
	SoilGrids &
	cfvo\_5-15cm\_Q0p5 &
	Fração volumétrica de fragmentos grosseiros (\textgreater 2 mm) em cm3/dm3 (vol‰) para centímetros 5 a 15 da superfície – predição para quantil 0,5 \\
	SoilGrids &
	cfvo\_5-15cm\_Q0p95 &
	Fração volumétrica de fragmentos grosseiros (\textgreater 2 mm) em cm3/dm3 (vol‰) para centímetros 5 a 15 da superfície – predição para quantil 0,95 \\
	SoilGrids &
	cfvo\_15-30cm\_mean &
	Fração volumétrica de fragmentos grosseiros (\textgreater 2 mm) em cm3/dm3 (vol‰) para centímetros 15 a 30 da superfície – média \\
	SoilGrids &
	cfvo\_15-30cm\_Q0p05 &
	Fração volumétrica de fragmentos grosseiros (\textgreater 2 mm) em cm3/dm3 (vol‰) para centímetros 15 a 30 da superfície – predição para quantil 0,05 \\
	SoilGrids &
	cfvo\_15-30cm\_Q0p5 &
	Fração volumétrica de fragmentos grosseiros (\textgreater 2 mm) em cm3/dm3 (vol‰) para centímetros 15 a 30 da superfície – predição para quantil 0,5 \\
	SoilGrids &
	cfvo\_15-30cm\_Q0p95 &
	Fração volumétrica de fragmentos grosseiros (\textgreater 2 mm) em cm3/dm3 (vol‰) para centímetros 15 a 30 da superfície – predição para quantil 0,95 \\
	SoilGrids &
	cfvo\_30-60cm\_mean &
	Fração volumétrica de fragmentos grosseiros (\textgreater 2 mm) em cm3/dm3 (vol‰) para centímetros 15 a 30 da superfície – média \\
	SoilGrids &
	cfvo\_30-60cm\_Q0p05 &
	Fração volumétrica de fragmentos grosseiros (\textgreater 2 mm) em cm3/dm3 (vol‰) para centímetros 30 a 60 da superfície – predição para quantil 0,05 \\
	SoilGrids &
	cfvo\_30-60cm\_Q0p5 &
	Fração volumétrica de fragmentos grosseiros (\textgreater 2 mm) em cm3/dm3 (vol‰) para centímetros 30 a 60 da superfície – predição para quantil 0,5 \\
	SoilGrids &
	cfvo\_30-60cm\_Q0p95 &
	Fração volumétrica de fragmentos grosseiros (\textgreater 2 mm) em cm3/dm3 (vol‰) para centímetros 30 a 60 da superfície – predição para quantil 0,95 \\
	SoilGrids &
	cfvo\_60-100cm\_mean &
	Fração volumétrica de fragmentos grosseiros (\textgreater 2 mm) em cm3/dm3 (vol‰) para centímetros 60 a 100 da superfície – média \\
	SoilGrids &
	cfvo\_60-100cm\_Q0p05 &
	Fração volumétrica de fragmentos grosseiros (\textgreater 2 mm) em cm3/dm3 (vol‰) para centímetros 60 a 100 da superfície – predição para quantil 0,05 \\
	SoilGrids &
	cfvo\_60-100cm\_Q0p5 &
	Fração volumétrica de fragmentos grosseiros (\textgreater 2 mm) em cm3/dm3 (vol‰) para centímetros 60 a 100 da superfície – predição para quantil 0,5 \\
	SoilGrids &
	cfvo\_60-100cm\_Q0p95 &
	Fração volumétrica de fragmentos grosseiros (\textgreater 2 mm) em cm3/dm3 (vol‰) para centímetros 60 a 100 da superfície – predição para quantil 0,95 \\
	SoilGrids &
	cfvo\_100-200cm\_mean &
	Fração volumétrica de fragmentos grosseiros (\textgreater 2 mm) em cm3/dm3 (vol‰) para centímetros 100 a 200 da superfície – média \\
	SoilGrids &
	cfvo\_100-200cm\_Q0p05 &
	Fração volumétrica de fragmentos grosseiros (\textgreater 2 mm) em cm3/dm3 (vol‰) para centímetros 100 a 200 da superfície – predição para quantil 0,05 \\
	SoilGrids &
	cfvo\_100-200cm\_Q0p5 &
	Fração volumétrica de fragmentos grosseiros (\textgreater 2 mm) em cm3/dm3 (vol‰) para centímetros 100 a 200 da superfície – predição para quantil 0,5 \\
	SoilGrids &
	cfvo\_100-200cm\_Q0p95 &
	Fração volumétrica de fragmentos grosseiros (\textgreater 2 mm) em cm3/dm3 (vol‰) para centímetros 100 a 200 da superfície – predição para quantil 0,95 \\
	SoilGrids &
	clay\_0-5cm\_mean &
	Proporção de partículas de argila (\textless 0,002 mm) na fração fina do solo em g/kg para centímetros 0 a 5 da superfície - média \\
	SoilGrids &
	clay\_0-5cm\_Q0p05 &
	Proporção de partículas de argila (\textless 0,002 mm) na fração fina do solo em g/kg para centímetros 0 a 5 da superfície – predição para quantil 0,05 \\
	SoilGrids &
	clay\_0-5cm\_Q0p5 &
	Proporção de partículas de argila (\textless 0,002 mm) na fração fina do solo em g/kg para centímetros 0 a 5 da superfície – predição para quantil 0,5 \\
	SoilGrids &
	clay\_0-5cm\_Q0p95 &
	Proporção de partículas de argila (\textless 0,002 mm) na fração fina do solo em g/kg para centímetros 0 a 5 da superfície – predição para quantil 0,95 \\
	SoilGrids &
	clay\_5-15cm\_mean &
	Proporção de partículas de argila (\textless 0,002 mm) na fração fina do solo em g/kg para centímetros 5 a 15 da superfície – média \\
	SoilGrids &
	clay\_5-15cm\_Q0p05 &
	Proporção de partículas de argila (\textless 0,002 mm) na fração fina do solo em g/kg para centímetros 5 a 15 da superfície – predição para quantil 0,05 \\
	SoilGrids &
	clay\_5-15cm\_Q0p5 &
	Proporção de partículas de argila (\textless 0,002 mm) na fração fina do solo em g/kg para centímetros 5 a 15 da superfície – predição para quantil 0,5 \\
	SoilGrids &
	clay\_5-15cm\_Q0p95 &
	Proporção de partículas de argila (\textless 0,002 mm) na fração fina do solo em g/kg para centímetros 5 a 15 da superfície – predição para quantil 0,95 \\
	SoilGrids &
	clay\_15-30cm\_mean &
	Proporção de partículas de argila (\textless 0,002 mm) na fração fina do solo em g/kg para centímetros 15 a 30 da superfície – média \\
	SoilGrids &
	clay\_15-30cm\_Q0p05 &
	Proporção de partículas de argila (\textless 0,002 mm) na fração fina do solo em g/kg para centímetros 15 a 30 da superfície – predição para quantil 0,05 \\
	SoilGrids &
	clay\_15-30cm\_Q0p5 &
	Proporção de partículas de argila (\textless 0,002 mm) na fração fina do solo em g/kg para centímetros 15 a 30 da superfície – predição para quantil 0,5 \\
	SoilGrids &
	clay\_15-30cm\_Q0p95 &
	Proporção de partículas de argila (\textless 0,002 mm) na fração fina do solo em g/kg para centímetros 15 a 30 da superfície – predição para quantil 0,95 \\
	SoilGrids &
	clay\_30-60cm\_mean &
	Proporção de partículas de argila (\textless 0,002 mm) na fração fina do solo em g/kg para centímetros 15 a 30 da superfície – média \\
	SoilGrids &
	clay\_30-60cm\_Q0p05 &
	Proporção de partículas de argila (\textless 0,002 mm) na fração fina do solo em g/kg para centímetros 30 a 60 da superfície – predição para quantil 0,05 \\
	SoilGrids &
	clay\_30-60cm\_Q0p5 &
	Proporção de partículas de argila (\textless 0,002 mm) na fração fina do solo em g/kg para centímetros 30 a 60 da superfície – predição para quantil 0,5 \\
	SoilGrids &
	clay\_30-60cm\_Q0p95 &
	Proporção de partículas de argila (\textless 0,002 mm) na fração fina do solo em g/kg para centímetros 30 a 60 da superfície – predição para quantil 0,95 \\
	SoilGrids &
	clay\_60-100cm\_mean &
	Proporção de partículas de argila (\textless 0,002 mm) na fração fina do solo em g/kg para centímetros 60 a 100 da superfície – média \\
	SoilGrids &
	clay\_60-100cm\_Q0p05 &
	Proporção de partículas de argila (\textless 0,002 mm) na fração fina do solo em g/kg para centímetros 60 a 100 da superfície – predição para quantil 0,05 \\
	SoilGrids &
	clay\_60-100cm\_Q0p5 &
	Proporção de partículas de argila (\textless 0,002 mm) na fração fina do solo em g/kg para centímetros 60 a 100 da superfície – predição para quantil 0,5 \\
	SoilGrids &
	clay\_60-100cm\_Q0p95 &
	Proporção de partículas de argila (\textless 0,002 mm) na fração fina do solo em g/kg para centímetros 60 a 100 da superfície – predição para quantil 0,95 \\
	SoilGrids &
	clay\_100-200cm\_mean &
	Proporção de partículas de argila (\textless 0,002 mm) na fração fina do solo em g/kg para centímetros 100 a 200 da superfície – média \\
	SoilGrids &
	clay\_100-200cm\_Q0p05 &
	Proporção de partículas de argila (\textless 0,002 mm) na fração fina do solo em g/kg para centímetros 100 a 200 da superfície – predição para quantil 0,05 \\
	SoilGrids &
	clay\_100-200cm\_Q0p5 &
	Proporção de partículas de argila (\textless 0,002 mm) na fração fina do solo em g/kg para centímetros 100 a 200 da superfície – predição para quantil 0,5 \\
	SoilGrids &
	clay\_100-200cm\_Q0p95 &
	Proporção de partículas de argila (\textless 0,002 mm) na fração fina do solo em g/kg para centímetros 100 a 200 da superfície – predição para quantil 0,95 \\
	SoilGrids &
	nitrogen\_0-5cm\_mean &
	Nitrogênio Total (N) cg/kg  para centímetros 0 a 5 da superfície - média \\
	SoilGrids &
	nitrogen\_0-5cm\_Q0p05 &
	Nitrogênio Total (N) cg/kg  para centímetros 0 a 5 da superfície – predição para quantil 0,05 \\
	SoilGrids &
	nitrogen\_0-5cm\_Q0p5 &
	Nitrogênio Total (N) cg/kg  para centímetros 0 a 5 da superfície – predição para quantil 0,5 \\
	SoilGrids &
	nitrogen\_0-5cm\_Q0p95 &
	Nitrogênio Total (N) cg/kg  para centímetros 0 a 5 da superfície – predição para quantil 0,95 \\
	SoilGrids &
	nitrogen\_5-15cm\_mean &
	Nitrogênio Total (N) cg/kg  para centímetros 5 a 15 da superfície – média \\
	SoilGrids &
	nitrogen\_5-15cm\_Q0p05 &
	Nitrogênio Total (N) cg/kg  para centímetros 5 a 15 da superfície – predição para quantil 0,05 \\
	SoilGrids &
	nitrogen\_5-15cm\_Q0p5 &
	Nitrogênio Total (N) cg/kg  para centímetros 5 a 15 da superfície – predição para quantil 0,5 \\
	SoilGrids &
	nitrogen\_5-15cm\_Q0p95 &
	Nitrogênio Total (N) cg/kg  para centímetros 5 a 15 da superfície – predição para quantil 0,95 \\
	SoilGrids &
	nitrogen\_15-30cm\_mean &
	Nitrogênio Total (N) cg/kg  para centímetros 15 a 30 da superfície – média \\
	SoilGrids &
	nitrogen\_15-30cm\_Q0p05 &
	Nitrogênio Total (N) cg/kg  para centímetros 15 a 30 da superfície – predição para quantil 0,05 \\
	SoilGrids &
	nitrogen\_15-30cm\_Q0p5 &
	Nitrogênio Total (N) cg/kg  para centímetros 15 a 30 da superfície – predição para quantil 0,5 \\
	SoilGrids &
	nitrogen\_15-30cm\_Q0p95 &
	Nitrogênio Total (N) cg/kg  para centímetros 15 a 30 da superfície – predição para quantil 0,95 \\
	SoilGrids &
	nitrogen\_30-60cm\_mean &
	Nitrogênio Total (N) cg/kg  para centímetros 15 a 30 da superfície – média \\
	SoilGrids &
	nitrogen\_30-60cm\_Q0p05 &
	Nitrogênio Total (N) cg/kg  para centímetros 30 a 60 da superfície – predição para quantil 0,05 \\
	SoilGrids &
	nitrogen\_30-60cm\_Q0p5 &
	Nitrogênio Total (N) cg/kg  para centímetros 30 a 60 da superfície – predição para quantil 0,5 \\
	SoilGrids &
	nitrogen\_30-60cm\_Q0p95 &
	Nitrogênio Total (N) cg/kg  para centímetros 30 a 60 da superfície – predição para quantil 0,95 \\
	SoilGrids &
	nitrogen\_60-100cm\_mean &
	Nitrogênio Total (N) cg/kg  para centímetros 60 a 100 da superfície – média \\
	SoilGrids &
	nitrogen\_60-100cm\_Q0p05 &
	Nitrogênio Total (N) cg/kg  para centímetros 60 a 100 da superfície – predição para quantil 0,05 \\
	SoilGrids &
	nitrogen\_60-100cm\_Q0p5 &
	Nitrogênio Total (N) cg/kg  para centímetros 60 a 100 da superfície – predição para quantil 0,5 \\
	SoilGrids &
	nitrogen\_60-100cm\_Q0p95 &
	Nitrogênio Total (N) cg/kg  para centímetros 60 a 100 da superfície – predição para quantil 0,95 \\
	SoilGrids &
	nitrogen\_100-200cm\_mean &
	Nitrogênio Total (N) cg/kg  para centímetros 100 a 200 da superfície – média \\
	SoilGrids &
	nitrogen\_100-200cm\_Q0p05 &
	Nitrogênio Total (N) cg/kg  para centímetros 100 a 200 da superfície – predição para quantil 0,05 \\
	SoilGrids &
	nitrogen\_100-200cm\_Q0p5 &
	Nitrogênio Total (N) cg/kg  para centímetros 100 a 200 da superfície – predição para quantil 0,5 \\
	SoilGrids &
	nitrogen\_100-200cm\_Q0p95 &
	Nitrogênio Total (N) cg/kg  para centímetros 100 a 200 da superfície – predição para quantil 0,95 \\
	SoilGrids &
	ocd\_0-5cm\_mean &
	Densidade de carbono orgânico em hg/m³ para centímetros 0 a 5 da superfície - média \\
	SoilGrids &
	ocd\_0-5cm\_Q0p05 &
	Densidade de carbono orgânico em hg/m³ para centímetros 0 a 5 da superfície – predição para quantil 0,05 \\
	SoilGrids &
	ocd\_0-5cm\_Q0p5 &
	Densidade de carbono orgânico em hg/m³ para centímetros 0 a 5 da superfície – predição para quantil 0,5 \\
	SoilGrids &
	ocd\_0-5cm\_Q0p95 &
	Densidade de carbono orgânico em hg/m³ para centímetros 0 a 5 da superfície – predição para quantil 0,95 \\
	SoilGrids &
	ocd\_5-15cm\_mean &
	Densidade de carbono orgânico em hg/m³ para centímetros 5 a 15 da superfície – média \\
	SoilGrids &
	ocd\_5-15cm\_Q0p05 &
	Densidade de carbono orgânico em hg/m³ para centímetros 5 a 15 da superfície – predição para quantil 0,05 \\
	SoilGrids &
	ocd\_5-15cm\_Q0p5 &
	Densidade de carbono orgânico em hg/m³ para centímetros 5 a 15 da superfície – predição para quantil 0,5 \\
	SoilGrids &
	ocd\_5-15cm\_Q0p95 &
	Densidade de carbono orgânico em hg/m³ para centímetros 5 a 15 da superfície – predição para quantil 0,95 \\
	SoilGrids &
	ocd\_15-30cm\_mean &
	Densidade de carbono orgânico em hg/m³ para centímetros 15 a 30 da superfície – média \\
	SoilGrids &
	ocd\_15-30cm\_Q0p05 &
	Densidade de carbono orgânico em hg/m³ para centímetros 15 a 30 da superfície – predição para quantil 0,05 \\
	SoilGrids &
	ocd\_15-30cm\_Q0p5 &
	Densidade de carbono orgânico em hg/m³ para centímetros 15 a 30 da superfície – predição para quantil 0,5 \\
	SoilGrids &
	ocd\_15-30cm\_Q0p95 &
	Densidade de carbono orgânico em hg/m³ para centímetros 15 a 30 da superfície – predição para quantil 0,95 \\
	SoilGrids &
	ocd\_30-60cm\_mean &
	Densidade de carbono orgânico em hg/m³ para centímetros 15 a 30 da superfície – média \\
	SoilGrids &
	ocd\_30-60cm\_Q0p05 &
	Densidade de carbono orgânico em hg/m³ para centímetros 30 a 60 da superfície – predição para quantil 0,05 \\
	SoilGrids &
	ocd\_30-60cm\_Q0p5 &
	Densidade de carbono orgânico em hg/m³ para centímetros 30 a 60 da superfície – predição para quantil 0,5 \\
	SoilGrids &
	ocd\_30-60cm\_Q0p95 &
	Densidade de carbono orgânico em hg/m³ para centímetros 30 a 60 da superfície – predição para quantil 0,95 \\
	SoilGrids &
	ocd\_60-100cm\_mean &
	Densidade de carbono orgânico em hg/m³ para centímetros 60 a 100 da superfície – média \\
	SoilGrids &
	ocd\_60-100cm\_Q0p05 &
	Densidade de carbono orgânico em hg/m³ para centímetros 60 a 100 da superfície – predição para quantil 0,05 \\
	SoilGrids &
	ocd\_60-100cm\_Q0p5 &
	Densidade de carbono orgânico em hg/m³ para centímetros 60 a 100 da superfície – predição para quantil 0,5 \\
	SoilGrids &
	ocd\_60-100cm\_Q0p95 &
	Densidade de carbono orgânico em hg/m³ para centímetros 60 a 100 da superfície – predição para quantil 0,95 \\
	SoilGrids &
	ocd\_100-200cm\_mean &
	Densidade de carbono orgânico em hg/m³ para centímetros 100 a 200 da superfície – média \\
	SoilGrids &
	ocd\_100-200cm\_Q0p05 &
	Densidade de carbono orgânico em hg/m³ para centímetros 100 a 200 da superfície – predição para quantil 0,05 \\
	SoilGrids &
	ocd\_100-200cm\_Q0p5 &
	Densidade de carbono orgânico em hg/m³ para centímetros 100 a 200 da superfície – predição para quantil 0,5 \\
	SoilGrids &
	ocd\_100-200cm\_Q0p95 &
	Densidade de carbono orgânico em hg/m³ para centímetros 100 a 200 da superfície – predição para quantil 0,95 \\
	SoilGrids &
	ocs\_0-30cm\_mean &
	Estoques de carbono orgânico em t/ha para centímetros 0 a 30 da superfície - média \\
	SoilGrids &
	ocs\_0-30cm\_Q0p05 &
	Estoques de carbono orgânico em t/ha para centímetros 0 a 30 da superfície – predição para quantil 0,05 \\
	SoilGrids &
	ocs\_0-30cm\_Q0p5 &
	Estoques de carbono orgânico em t/ha para centímetros 0 a 30 da superfície – predição para quantil 0,5 \\
	SoilGrids &
	ocs\_0-30cm\_Q0p95 &
	Estoques de carbono orgânico em t/ha para centímetros 0 a 30 da superfície – predição para quantil 0,95 \\
	SoilGrids &
	phh2o\_0-5cm\_mean &
	pH do solo em pHx10 para centímetros 0 a 5 da superfície - média \\
	SoilGrids &
	phh2o\_0-5cm\_Q0p05 &
	pH do solo em pHx10 para centímetros 0 a 5 da superfície – predição para quantil 0,05 \\
	SoilGrids &
	phh2o\_0-5cm\_Q0p5 &
	pH do solo em pHx10 para centímetros 0 a 5 da superfície – predição para quantil 0,5 \\
	SoilGrids &
	phh2o\_0-5cm\_Q0p95 &
	pH do solo em pHx10 para centímetros 0 a 5 da superfície – predição para quantil 0,95 \\
	SoilGrids &
	phh2o\_5-15cm\_mean &
	pH do solo em pHx10 para centímetros 5 a 15 da superfície – média \\
	SoilGrids &
	phh2o\_5-15cm\_Q0p05 &
	pH do solo em pHx10 para centímetros 5 a 15 da superfície – predição para quantil 0,05 \\
	SoilGrids &
	phh2o\_5-15cm\_Q0p5 &
	pH do solo em pHx10 para centímetros 5 a 15 da superfície – predição para quantil 0,5 \\
	SoilGrids &
	phh2o\_5-15cm\_Q0p95 &
	pH do solo em pHx10 para centímetros 5 a 15 da superfície – predição para quantil 0,95 \\
	SoilGrids &
	phh2o\_15-30cm\_mean &
	pH do solo em pHx10 para centímetros 15 a 30 da superfície – média \\
	SoilGrids &
	phh2o\_15-30cm\_Q0p05 &
	pH do solo em pHx10 para centímetros 15 a 30 da superfície – predição para quantil 0,05 \\
	SoilGrids &
	phh2o\_15-30cm\_Q0p5 &
	pH do solo em pHx10 para centímetros 15 a 30 da superfície – predição para quantil 0,5 \\
	SoilGrids &
	phh2o\_15-30cm\_Q0p95 &
	pH do solo em pHx10 para centímetros 15 a 30 da superfície – predição para quantil 0,95 \\
	SoilGrids &
	phh2o\_30-60cm\_mean &
	pH do solo em pHx10 para centímetros 15 a 30 da superfície – média \\
	SoilGrids &
	phh2o\_30-60cm\_Q0p05 &
	pH do solo em pHx10 para centímetros 30 a 60 da superfície – predição para quantil 0,05 \\
	SoilGrids &
	phh2o\_30-60cm\_Q0p5 &
	pH do solo em pHx10 para centímetros 30 a 60 da superfície – predição para quantil 0,5 \\
	SoilGrids &
	phh2o\_30-60cm\_Q0p95 &
	pH do solo em pHx10 para centímetros 30 a 60 da superfície – predição para quantil 0,95 \\
	SoilGrids &
	phh2o\_60-100cm\_mean &
	pH do solo em pHx10 para centímetros 60 a 100 da superfície – média \\
	SoilGrids &
	phh2o\_60-100cm\_Q0p05 &
	pH do solo em pHx10 para centímetros 60 a 100 da superfície – predição para quantil 0,05 \\
	SoilGrids &
	phh2o\_60-100cm\_Q0p5 &
	pH do solo em pHx10 para centímetros 60 a 100 da superfície – predição para quantil 0,5 \\
	SoilGrids &
	phh2o\_60-100cm\_Q0p95 &
	pH do solo em pHx10 para centímetros 60 a 100 da superfície – predição para quantil 0,95 \\
	SoilGrids &
	phh2o\_100-200cm\_mean &
	pH do solo em pHx10 para centímetros 100 a 200 da superfície – média \\
	SoilGrids &
	phh2o\_100-200cm\_Q0p05 &
	pH do solo em pHx10 para centímetros 100 a 200 da superfície – predição para quantil 0,05 \\
	SoilGrids &
	phh2o\_100-200cm\_Q0p5 &
	pH do solo em pHx10 para centímetros 100 a 200 da superfície – predição para quantil 0,5 \\
	SoilGrids &
	phh2o\_100-200cm\_Q0p95 &
	pH do solo em pHx10 para centímetros 100 a 200 da superfície – predição para quantil 0,95 \\
	SoilGrids &
	sand\_0-5cm\_mean &
	Proportion of sand particles (\textgreater 0.05 mm) in the fine earth fraction g/kg para centímetros 0 a 5 da superfície - média \\
	SoilGrids &
	sand\_0-5cm\_Q0p05 &
	Proportion of sand particles (\textgreater 0.05 mm) in the fine earth fraction g/kg para centímetros 0 a 5 da superfície – predição para quantil 0,05 \\
	SoilGrids &
	sand\_0-5cm\_Q0p5 &
	Proportion of sand particles (\textgreater 0.05 mm) in the fine earth fraction g/kg para centímetros 0 a 5 da superfície – predição para quantil 0,5 \\
	SoilGrids &
	sand\_0-5cm\_Q0p95 &
	Proportion of sand particles (\textgreater 0.05 mm) in the fine earth fraction g/kg para centímetros 0 a 5 da superfície – predição para quantil 0,95 \\
	SoilGrids &
	sand\_5-15cm\_mean &
	Proportion of sand particles (\textgreater 0.05 mm) in the fine earth fraction g/kg para centímetros 5 a 15 da superfície – média \\
	SoilGrids &
	sand\_5-15cm\_Q0p05 &
	Proportion of sand particles (\textgreater 0.05 mm) in the fine earth fraction g/kg para centímetros 5 a 15 da superfície – predição para quantil 0,05 \\
	SoilGrids &
	sand\_5-15cm\_Q0p5 &
	Proportion of sand particles (\textgreater 0.05 mm) in the fine earth fraction g/kg para centímetros 5 a 15 da superfície – predição para quantil 0,5 \\
	SoilGrids &
	sand\_5-15cm\_Q0p95 &
	Proportion of sand particles (\textgreater 0.05 mm) in the fine earth fraction g/kg para centímetros 5 a 15 da superfície – predição para quantil 0,95 \\
	SoilGrids &
	sand\_15-30cm\_mean &
	Proportion of sand particles (\textgreater 0.05 mm) in the fine earth fraction g/kg para centímetros 15 a 30 da superfície – média \\
	SoilGrids &
	sand\_15-30cm\_Q0p05 &
	Proportion of sand particles (\textgreater 0.05 mm) in the fine earth fraction g/kg para centímetros 15 a 30 da superfície – predição para quantil 0,05 \\
	SoilGrids &
	sand\_15-30cm\_Q0p5 &
	Proportion of sand particles (\textgreater 0.05 mm) in the fine earth fraction g/kg para centímetros 15 a 30 da superfície – predição para quantil 0,5 \\
	SoilGrids &
	sand\_15-30cm\_Q0p95 &
	Proportion of sand particles (\textgreater 0.05 mm) in the fine earth fraction g/kg para centímetros 15 a 30 da superfície – predição para quantil 0,95 \\
	SoilGrids &
	sand\_30-60cm\_mean &
	Proportion of sand particles (\textgreater 0.05 mm) in the fine earth fraction g/kg para centímetros 15 a 30 da superfície – média \\
	SoilGrids &
	sand\_30-60cm\_Q0p05 &
	Proportion of sand particles (\textgreater 0.05 mm) in the fine earth fraction g/kg para centímetros 30 a 60 da superfície – predição para quantil 0,05 \\
	SoilGrids &
	sand\_30-60cm\_Q0p5 &
	Proportion of sand particles (\textgreater 0.05 mm) in the fine earth fraction g/kg para centímetros 30 a 60 da superfície – predição para quantil 0,5 \\
	SoilGrids &
	sand\_30-60cm\_Q0p95 &
	Proportion of sand particles (\textgreater 0.05 mm) in the fine earth fraction g/kg para centímetros 30 a 60 da superfície – predição para quantil 0,95 \\
	SoilGrids &
	sand\_60-100cm\_mean &
	Proportion of sand particles (\textgreater 0.05 mm) in the fine earth fraction g/kg para centímetros 60 a 100 da superfície – média \\
	SoilGrids &
	sand\_60-100cm\_Q0p05 &
	Proportion of sand particles (\textgreater 0.05 mm) in the fine earth fraction g/kg para centímetros 60 a 100 da superfície – predição para quantil 0,05 \\
	SoilGrids &
	sand\_60-100cm\_Q0p5 &
	Proportion of sand particles (\textgreater 0.05 mm) in the fine earth fraction g/kg para centímetros 60 a 100 da superfície – predição para quantil 0,5 \\
	SoilGrids &
	sand\_60-100cm\_Q0p95 &
	Proportion of sand particles (\textgreater 0.05 mm) in the fine earth fraction g/kg para centímetros 60 a 100 da superfície – predição para quantil 0,95 \\
	SoilGrids &
	sand\_100-200cm\_mean &
	Proportion of sand particles (\textgreater 0.05 mm) in the fine earth fraction g/kg para centímetros 100 a 200 da superfície – média \\
	SoilGrids &
	sand\_100-200cm\_Q0p05 &
	Proportion of sand particles (\textgreater 0.05 mm) in the fine earth fraction g/kg para centímetros 100 a 200 da superfície – predição para quantil 0,05 \\
	SoilGrids &
	sand\_100-200cm\_Q0p5 &
	Proportion of sand particles (\textgreater 0.05 mm) in the fine earth fraction g/kg para centímetros 100 a 200 da superfície – predição para quantil 0,5 \\
	SoilGrids &
	sand\_100-200cm\_Q0p95 &
	Proportion of sand particles (\textgreater 0.05 mm) in the fine earth fraction g/kg para centímetros 100 a 200 da superfície – predição para quantil 0,95 \\
	SoilGrids &
	silt\_0-5cm\_mean &
	Proporção de partículas de areia (\textgreater 0,05 mm) na fração fina do solo em g/kg para centímetros 0 a 5 da superfície - média \\
	SoilGrids &
	silt\_0-5cm\_Q0p05 &
	Proporção de partículas de areia (\textgreater 0,05 mm) na fração fina do solo em g/kg para centímetros 0 a 5 da superfície – predição para quantil 0,05 \\
	SoilGrids &
	silt\_0-5cm\_Q0p5 &
	Proporção de partículas de areia (\textgreater 0,05 mm) na fração fina do solo em g/kg para centímetros 0 a 5 da superfície – predição para quantil 0,5 \\
	SoilGrids &
	silt\_0-5cm\_Q0p95 &
	Proporção de partículas de areia (\textgreater 0,05 mm) na fração fina do solo em g/kg para centímetros 0 a 5 da superfície – predição para quantil 0,95 \\
	SoilGrids &
	silt\_5-15cm\_mean &
	Proporção de partículas de areia (\textgreater 0,05 mm) na fração fina do solo em g/kg para centímetros 5 a 15 da superfície – média \\
	SoilGrids &
	silt\_5-15cm\_Q0p05 &
	Proporção de partículas de areia (\textgreater 0,05 mm) na fração fina do solo em g/kg para centímetros 5 a 15 da superfície – predição para quantil 0,05 \\
	SoilGrids &
	silt\_5-15cm\_Q0p5 &
	Proporção de partículas de areia (\textgreater 0,05 mm) na fração fina do solo em g/kg para centímetros 5 a 15 da superfície – predição para quantil 0,5 \\
	SoilGrids &
	silt\_5-15cm\_Q0p95 &
	Proporção de partículas de areia (\textgreater 0,05 mm) na fração fina do solo em g/kg para centímetros 5 a 15 da superfície – predição para quantil 0,95 \\
	SoilGrids &
	silt\_15-30cm\_mean &
	Proporção de partículas de areia (\textgreater 0,05 mm) na fração fina do solo em g/kg para centímetros 15 a 30 da superfície – média \\
	SoilGrids &
	silt\_15-30cm\_Q0p05 &
	Proporção de partículas de areia (\textgreater 0,05 mm) na fração fina do solo em g/kg para centímetros 15 a 30 da superfície – predição para quantil 0,05 \\
	SoilGrids &
	silt\_15-30cm\_Q0p5 &
	Proporção de partículas de areia (\textgreater 0,05 mm) na fração fina do solo em g/kg para centímetros 15 a 30 da superfície – predição para quantil 0,5 \\
	SoilGrids &
	silt\_15-30cm\_Q0p95 &
	Proporção de partículas de areia (\textgreater 0,05 mm) na fração fina do solo em g/kg para centímetros 15 a 30 da superfície – predição para quantil 0,95 \\
	SoilGrids &
	silt\_30-60cm\_mean &
	Proporção de partículas de areia (\textgreater 0,05 mm) na fração fina do solo em g/kg para centímetros 15 a 30 da superfície – média \\
	SoilGrids &
	silt\_30-60cm\_Q0p05 &
	Proporção de partículas de areia (\textgreater 0,05 mm) na fração fina do solo em g/kg para centímetros 30 a 60 da superfície – predição para quantil 0,05 \\
	SoilGrids &
	silt\_30-60cm\_Q0p5 &
	Proporção de partículas de areia (\textgreater 0,05 mm) na fração fina do solo em g/kg para centímetros 30 a 60 da superfície – predição para quantil 0,5 \\
	SoilGrids &
	silt\_30-60cm\_Q0p95 &
	Proporção de partículas de areia (\textgreater 0,05 mm) na fração fina do solo em g/kg para centímetros 30 a 60 da superfície – predição para quantil 0,95 \\
	SoilGrids &
	silt\_60-100cm\_mean &
	Proporção de partículas de areia (\textgreater 0,05 mm) na fração fina do solo em g/kg para centímetros 60 a 100 da superfície – média \\
	SoilGrids &
	silt\_60-100cm\_Q0p05 &
	Proporção de partículas de areia (\textgreater 0,05 mm) na fração fina do solo em g/kg para centímetros 60 a 100 da superfície – predição para quantil 0,05 \\
	SoilGrids &
	silt\_60-100cm\_Q0p5 &
	Proporção de partículas de areia (\textgreater 0,05 mm) na fração fina do solo em g/kg para centímetros 60 a 100 da superfície – predição para quantil 0,5 \\
	SoilGrids &
	silt\_60-100cm\_Q0p95 &
	Proporção de partículas de areia (\textgreater 0,05 mm) na fração fina do solo em g/kg para centímetros 60 a 100 da superfície – predição para quantil 0,95 \\
	SoilGrids &
	silt\_100-200cm\_mean &
	Proporção de partículas de areia (\textgreater 0,05 mm) na fração fina do solo em g/kg para centímetros 100 a 200 da superfície – média \\
	SoilGrids &
	silt\_100-200cm\_Q0p05 &
	Proporção de partículas de areia (\textgreater 0,05 mm) na fração fina do solo em g/kg para centímetros 100 a 200 da superfície – predição para quantil 0,05 \\
	SoilGrids &
	silt\_100-200cm\_Q0p5 &
	Proporção de partículas de areia (\textgreater 0,05 mm) na fração fina do solo em g/kg para centímetros 100 a 200 da superfície – predição para quantil 0,5 \\
	SoilGrids &
	silt\_100-200cm\_Q0p95 &
	Proporção de partículas de areia (\textgreater 0,05 mm) na fração fina do solo em g/kg para centímetros 100 a 200 da superfície – predição para quantil 0,95 \\
	SoilGrids &
	soc\_0-5cm\_mean &
	Conteúdo de carbono orgânico do solo na fração fina do solo em dg/kg para centímetros 0 a 5 da superfície - média \\
	SoilGrids &
	soc\_0-5cm\_Q0p05 &
	Conteúdo de carbono orgânico do solo na fração fina do solo em dg/kg para centímetros 0 a 5 da superfície – predição para quantil 0,05 \\
	SoilGrids &
	soc\_0-5cm\_Q0p5 &
	Conteúdo de carbono orgânico do solo na fração fina do solo em dg/kg para centímetros 0 a 5 da superfície – predição para quantil 0,5 \\
	SoilGrids &
	soc\_0-5cm\_Q0p95 &
	Conteúdo de carbono orgânico do solo na fração fina do solo em dg/kg para centímetros 0 a 5 da superfície – predição para quantil 0,95 \\
	SoilGrids &
	soc\_5-15cm\_mean &
	Conteúdo de carbono orgânico do solo na fração fina do solo em dg/kg para centímetros 5 a 15 da superfície – média \\
	SoilGrids &
	soc\_5-15cm\_Q0p05 &
	Conteúdo de carbono orgânico do solo na fração fina do solo em dg/kg para centímetros 5 a 15 da superfície – predição para quantil 0,05 \\
	SoilGrids &
	soc\_5-15cm\_Q0p5 &
	Conteúdo de carbono orgânico do solo na fração fina do solo em dg/kg para centímetros 5 a 15 da superfície – predição para quantil 0,5 \\
	SoilGrids &
	soc\_5-15cm\_Q0p95 &
	Conteúdo de carbono orgânico do solo na fração fina do solo em dg/kg para centímetros 5 a 15 da superfície – predição para quantil 0,95 \\
	SoilGrids &
	soc\_15-30cm\_mean &
	Conteúdo de carbono orgânico do solo na fração fina do solo em dg/kg para centímetros 15 a 30 da superfície – média \\
	SoilGrids &
	soc\_15-30cm\_Q0p05 &
	Conteúdo de carbono orgânico do solo na fração fina do solo em dg/kg para centímetros 15 a 30 da superfície – predição para quantil 0,05 \\
	SoilGrids &
	soc\_15-30cm\_Q0p5 &
	Conteúdo de carbono orgânico do solo na fração fina do solo em dg/kg para centímetros 15 a 30 da superfície – predição para quantil 0,5 \\
	SoilGrids &
	soc\_15-30cm\_Q0p95 &
	Conteúdo de carbono orgânico do solo na fração fina do solo em dg/kg para centímetros 15 a 30 da superfície – predição para quantil 0,95 \\
	SoilGrids &
	soc\_30-60cm\_mean &
	Conteúdo de carbono orgânico do solo na fração fina do solo em dg/kg para centímetros 15 a 30 da superfície – média \\
	SoilGrids &
	soc\_30-60cm\_Q0p05 &
	Conteúdo de carbono orgânico do solo na fração fina do solo em dg/kg para centímetros 30 a 60 da superfície – predição para quantil 0,05 \\
	SoilGrids &
	soc\_30-60cm\_Q0p5 &
	Conteúdo de carbono orgânico do solo na fração fina do solo em dg/kg para centímetros 30 a 60 da superfície – predição para quantil 0,5 \\
	SoilGrids &
	soc\_30-60cm\_Q0p95 &
	Conteúdo de carbono orgânico do solo na fração fina do solo em dg/kg para centímetros 30 a 60 da superfície – predição para quantil 0,95 \\
	SoilGrids &
	soc\_60-100cm\_mean &
	Conteúdo de carbono orgânico do solo na fração fina do solo em dg/kg para centímetros 60 a 100 da superfície – média \\
	SoilGrids &
	soc\_60-100cm\_Q0p05 &
	Conteúdo de carbono orgânico do solo na fração fina do solo em dg/kg para centímetros 60 a 100 da superfície – predição para quantil 0,05 \\
	SoilGrids &
	soc\_60-100cm\_Q0p5 &
	Conteúdo de carbono orgânico do solo na fração fina do solo em dg/kg para centímetros 60 a 100 da superfície – predição para quantil 0,5 \\
	SoilGrids &
	soc\_60-100cm\_Q0p95 &
	Conteúdo de carbono orgânico do solo na fração fina do solo em dg/kg para centímetros 60 a 100 da superfície – predição para quantil 0,95 \\
	SoilGrids &
	soc\_100-200cm\_mean &
	Conteúdo de carbono orgânico do solo na fração fina do solo em dg/kg para centímetros 100 a 200 da superfície – média \\
	SoilGrids &
	soc\_100-200cm\_Q0p05 &
	Conteúdo de carbono orgânico do solo na fração fina do solo em dg/kg para centímetros 100 a 200 da superfície – predição para quantil 0,05 \\
	SoilGrids &
	soc\_100-200cm\_Q0p5 &
	Conteúdo de carbono orgânico do solo na fração fina do solo em dg/kg para centímetros 100 a 200 da superfície – predição para quantil 0,5 \\
	SoilGrids &
	soc\_100-200cm\_Q0p95 &
	Conteúdo de carbono orgânico do solo na fração fina do solo em dg/kg para centímetros 100 a 200 da superfície – predição para quantil 0,95 \\
	NasaPower &
	T2M 2010a2019ANN &
	Média anual da temperatura média a 2 metros acima do solo para o período de 2010 a 2019. \\
	NasaPower &
	RH2M 2010a2019ANN &
	Média anual da umidade relativa a 2 metros acima do solo para o período de 2010 a 2019. \\
	NasaPower &
	WS2M 2010a2019ANN &
	Média anual da velocidade do vento a 2 metros acima do solo para o período de 2010 a 2019. \\
	NasaPower &
	T2MDEW 2010a2019ANN &
	Média anual da temperatura do ponto de orvalho a 2 metros acima do solo para o período de 2010 a 2019. \\
	NasaPower &
	T2M MAX 2010a2019ANN &
	Média anual da temperatura máxima a 2 metros acima do solo para o período de 2010 a 2019. \\
	NasaPower &
	T2M MIN 2010a2019ANN &
	Média anual da temperatura mínima a 2 metros acima do solo para o período de 2010 a 2019. \\
	NasaPower &
	PRECTOTCORR 2010a2019ANN &
	Precipitação total corrigida para o período de 2010 a 2019. O "CORR" indica que esta medição foi corrigida ou ajustada de alguma forma. \\
	NasaPower &
	ALLSKY SFC LW DWN 2010a2019ANN &
	Radiação descendente de onda longa na superfície sob todas as condições de céu para o período de 2010 a 2019. Média ao longo da década. \\
	NasaPower &
	ALLSKY SFC SW DWN 2010a2019ANN &
	Radiação descendente de onda curta na superfície sob todas as condições de céu para o período de 2010 a 2019. Média ao longo da década. \\
	NasaPower &
	alt 2010a2019ANN &
	Altitude média do local para o qual os dados foram registrados durante o período de 2010 a 2019. \\
	NasaPower &
	T2M 2010a2019APR &
	Temperatura média a 2 metros acima do solo durante o mês de abril para cada ano de 2010 a 2019. \\
	NasaPower &
	RH2M 2010a2019APR &
	Umidade relativa média a 2 metros acima do solo durante abril para os anos especificados. \\
	NasaPower &
	WS2M 2010a2019APR &
	Velocidade média do vento a 2 metros acima do solo durante abril ao longo do período de 2010 a 2019. \\
	NasaPower &
	T2MDEW 2010a2019APR &
	Temperatura média do ponto de orvalho a 2 metros durante abril para cada ano no intervalo especificado. O ponto de orvalho é um indicador de umidade atmosférica. \\
	NasaPower &
	T2M MAX 2010a2019APR &
	Temperatura máxima média a 2 metros acima do solo durante abril para os anos de 2010 a 2019. \\
	NasaPower &
	T2M MIN 2010a2019APR &
	Temperatura mínima média a 2 metros acima do solo durante abril durante o mesmo período. \\
	NasaPower &
	PRECTOTCORR 2010a2019APR &
	Precipitação total corrigida registrada durante abril de cada ano de 2010 a 2019. As correções normalmente ajustam diversos fatores para garantir precisão. \\
	NasaPower &
	ALLSKY SFC LW DWN 2010a2019APR &
	Radiação descendente de onda longa na superfície sob todas as condições de céu durante abril de cada ano na década. Mede a radiação de onda longa que alcança a superfície terrestre. \\
	NasaPower &
	ALLSKY SFC SW DWN 2010a2019APR &
	Radiação descendente de onda curta na superfície sob todas as condições de céu durante abril para cada ano de 2010 a 2019. Mede a radiação de onda curta (como a luz solar) que alcança a superfície. \\
	NasaPower &
	alt 2010a2019APR &
	Representa a altitude média do(s) local(is) de onde os dados de abril foram coletados ao longo desses anos. \\
	NasaPower &
	T2M 2010a2019AUG &
	Temperatura média a 2 metros acima do solo durante o mês de agosto para cada ano de 2010 a 2019. \\
	NasaPower &
	RH2M 2010a2019AUG &
	Umidade relativa média a 2 metros acima do solo registrada durante agosto ao longo dos anos especificados. \\
	NasaPower &
	WS2M 2010a2019AUG &
	Velocidade média do vento a 2 metros acima do solo medida durante agosto para os anos de 2010 a 2019. \\
	NasaPower &
	T2MDEW 2010a2019AUG &
	Temperatura média do ponto de orvalho a 2 metros durante agosto para os anos de 2010 a 2019. O ponto de orvalho dá uma indicação do conteúdo de umidade no ar. \\
	NasaPower &
	T2M MAX 2010a2019AUG &
	Temperatura máxima média a 2 metros acima do solo durante agosto para a década de interesse. \\
	NasaPower &
	T2M MIN 2010a2019AUG &
	Temperatura mínima média a 2 metros acima do solo durante agosto durante o mesmo período. \\
	NasaPower &
	PRECTOTCORR 2010a2019AUG &
	Precipitação total corrigida que ocorreu durante agosto de cada ano de 2010 a 2019. "Corrigida" implica ajustes para anomalias ou erros de medição. \\
	NasaPower &
	ALLSKY SFC LW DWN 2010a2019AUG &
	Radiação descendente de onda longa na superfície sob todas as condições de céu durante o mês de agosto ao longo dos anos de 2010 a 2019. Reflete a radiação de onda longa que alcança a superfície terrestre. \\
	NasaPower &
	ALLSKY SFC SW DWN 2010a2019AUG &
	Radiação descendente de onda curta na superfície sob todas as condições de céu durante agosto para a década em questão. \\
	NasaPower &
	alt 2010a2019AUG &
	Representa a altitude média do(s) local(is) de onde os dados de agosto foram coletados ao longo desses anos. \\
	NasaPower &
	T2M 2010a2019DEC &
	Temperatura média a 2 metros acima do solo durante o mês de dezembro para cada ano de 2010 a 2019. \\
	NasaPower &
	RH2M 2010a2019DEC &
	Umidade relativa média a 2 metros acima do solo durante dezembro para o período especificado. \\
	NasaPower &
	WS2M 2010a2019DEC &
	Velocidade média do vento a 2 metros acima do solo durante dezembro ao longo dos anos de 2010 a 2019. \\
	NasaPower &
	T2MDEW 2010a2019DEC &
	Temperatura média do ponto de orvalho a 2 metros durante dezembro para os anos de 2010 a 2019. A temperatura do ponto de orvalho é uma medida de umidade atmosférica. \\
	NasaPower &
	T2M MAX 2010a2019DEC &
	Temperatura máxima média a 2 metros acima do solo durante dezembro para a década em questão. \\
	NasaPower &
	T2M MIN 2010a2019DEC &
	Temperatura mínima média a 2 metros acima do solo durante dezembro durante o mesmo período. \\
	NasaPower &
	PRECTOTCORR 2010a2019DEC &
	Precipitação total corrigida observada durante dezembro de cada ano de 2010 a 2019. As correções normalmente envolvem ajustes nos dados brutos para eliminar viéses ou erros conhecidos. \\
	NasaPower &
	ALLSKY SFC LW DWN 2010a2019DEC &
	Radiação descendente de onda longa na superfície sob todas as condições de céu para dezembro ao longo dos anos de 2010 a 2019. Reflete a radiação de onda longa que alcança a superfície terrestre. \\
	NasaPower &
	ALLSKY SFC SW DWN 2010a2019DEC &
	Radiação descendente de onda curta na superfície sob todas as condições de céu durante dezembro para cada um dos anos de 2010 a 2019. \\
	NasaPower &
	alt 2010a2019DEC &
	Representa a altitude média do(s) local(is) de onde os dados de dezembro foram coletados ao longo desses anos. \\
	NasaPower &
	T2M 2010a2019FEB &
	Temperatura média a 2 metros acima do solo durante o mês de fevereiro para cada ano de 2010 a 2019. \\
	NasaPower &
	RH2M 2010a2019FEB &
	Umidade relativa média a 2 metros acima do solo registrada durante fevereiro ao longo dos anos especificados. \\
	NasaPower &
	WS2M 2010a2019FEB &
	Velocidade média do vento a 2 metros acima do solo medida durante fevereiro para os anos de 2010 a 2019. \\
	NasaPower &
	T2MDEW 2010a2019FEB &
	Temperatura média do ponto de orvalho a 2 metros durante fevereiro para os anos de 2010 a 2019. A temperatura do ponto de orvalho é um indicador importante de umidade atmosférica. \\
	NasaPower &
	T2M MAX 2010a2019FEB &
	Temperatura máxima média a 2 metros acima do solo durante fevereiro para os anos de 2010 a 2019. \\
	NasaPower &
	T2M MIN 2010a2019FEB &
	Temperatura mínima média a 2 metros acima do solo durante fevereiro durante o período especificado. \\
	NasaPower &
	PRECTOTCORR 2010a2019FEB &
	Precipitação total corrigida registrada durante fevereiro de cada ano de 2010 a 2019. "Corrigida" implica que os dados foram ajustados para corrigir erros ou viéses de medição. \\
	NasaPower &
	ALLSKY SFC LW DWN 2010a2019FEB &
	Radiação descendente de onda longa na superfície sob todas as condições de céu para o mês de fevereiro ao longo dos anos de 2010 a 2019. Reflete a radiação de onda longa que alcança a superfície terrestre. \\
	NasaPower &
	ALLSKY SFC SW DWN 2010a2019FEB &
	Radiação descendente de onda curta na superfície sob todas as condições de céu durante fevereiro para cada um dos anos de 2010 a 2019. \\
	NasaPower &
	alt 2010a2019FEB &
	Representa a altitude média do(s) local(is) de onde os dados de fevereiro foram coletados ao longo desses anos. \\
	NasaPower &
	T2M 2010a2019JAN &
	Temperatura média a 2 metros acima do solo durante o mês de janeiro para cada ano de 2010 a 2019. \\
	NasaPower &
	RH2M 2010a2019JAN &
	Umidade relativa média a 2 metros acima do solo durante janeiro para o período especificado. \\
	NasaPower &
	WS2M 2010a2019JAN &
	Velocidade média do vento a 2 metros acima do solo medida durante janeiro para os anos de 2010 a 2019. \\
	NasaPower &
	T2MDEW 2010a2019JAN &
	Temperatura média do ponto de orvalho a 2 metros durante janeiro para os anos de 2010 a 2019. O ponto de orvalho fornece uma medida da quantidade de umidade no ar. \\
	NasaPower &
	T2M MAX 2010a2019JAN &
	Temperatura máxima média a 2 metros acima do solo durante janeiro para a década de interesse. \\
	NasaPower &
	T2M MIN 2010a2019JAN &
	Temperatura mínima média a 2 metros acima do solo durante janeiro durante o período especificado. \\
	NasaPower &
	PRECTOTCORR 2010a2019JAN &
	Precipitação total corrigida observada durante janeiro de cada ano de 2010 a 2019. O aspecto "corrigido" normalmente aborda quaisquer anomalias ou erros de medição. \\
	NasaPower &
	ALLSKY SFC LW DWN 2010a2019JAN &
	Radiação descendente de onda longa na superfície sob todas as condições de céu para janeiro ao longo dos anos de 2010 a 2019. Reflete a radiação de onda longa que alcança a superfície terrestre. \\
	NasaPower &
	ALLSKY SFC SW DWN 2010a2019JAN &
	Radiação descendente de onda curta na superfície sob todas as condições de céu durante janeiro para cada um dos anos de 2010 a 2019. \\
	NasaPower &
	alt 2010a2019JAN &
	Representa a altitude média do(s) local(is) de onde os dados de janeiro foram coletados ao longo desses anos. \\
	NasaPower &
	T2M 2010a2019JUL &
	Temperatura média a 2 metros acima do solo durante o mês de julho para cada ano de 2010 a 2019. \\
	NasaPower &
	RH2M 2010a2019JUL &
	Umidade relativa média a 2 metros acima do solo durante julho para o período especificado. \\
	NasaPower &
	WS2M 2010a2019JUL &
	Velocidade média do vento a 2 metros acima do solo medida durante julho para os anos de 2010 a 2019. \\
	NasaPower &
	T2MDEW 2010a2019JUL &
	Temperatura média do ponto de orvalho a 2 metros durante julho para os anos de 2010 a 2019. A temperatura do ponto de orvalho é uma medida de umidade atmosférica. \\
	NasaPower &
	T2M MAX 2010a2019JUL &
	Temperatura máxima média a 2 metros acima do solo durante julho para os anos de 2010 a 2019. \\
	NasaPower &
	T2M MIN 2010a2019JUL &
	Temperatura mínima média a 2 metros acima do solo durante julho durante o período especificado. \\
	NasaPower &
	PRECTOTCORR 2010a2019JUL &
	Precipitação total corrigida registrada durante julho de cada ano de 2010 a 2019. O termo "corrigida" sugere que os dados foram ajustados para contabilizar erros ou viéses de medição. \\
	NasaPower &
	ALLSKY SFC LW DWN 2010a2019JUL &
	Radiação descendente de onda longa na superfície sob todas as condições de céu para julho ao longo dos anos de 2010 a 2019. Reflete a radiação de onda longa que alcança a superfície terrestre. \\
	NasaPower &
	ALLSKY SFC SW DWN 2010a2019JUL &
	Radiação descendente de onda curta na superfície sob todas as condições de céu durante julho para cada um dos anos de 2010 a 2019. \\
	NasaPower &
	alt 2010a2019JUL &
	Representa a altitude média do(s) local(is) de onde os dados de julho foram coletados ao longo desses anos. \\
	NasaPower &
	T2M 2010a2019JUN &
	Temperatura média a 2 metros acima do solo durante o mês de junho para cada ano de 2010 a 2019. \\
	NasaPower &
	RH2M 2010a2019JUN &
	Umidade relativa média a 2 metros acima do solo durante junho para o período especificado. \\
	NasaPower &
	WS2M 2010a2019JUN &
	Velocidade média do vento a 2 metros acima do solo medida durante junho para os anos de 2010 a 2019. \\
	NasaPower &
	T2MDEW 2010a2019JUN &
	Temperatura média do ponto de orvalho a 2 metros durante junho para os anos de 2010 a 2019. O ponto de orvalho é um indicador da quantidade de umidade no ar. \\
	NasaPower &
	T2M MAX 2010a2019JUN &
	Temperatura máxima média a 2 metros acima do solo durante junho para os anos de 2010 a 2019. \\
	NasaPower &
	T2M MIN 2010a2019JUN &
	Temperatura mínima média a 2 metros acima do solo durante junho durante o período especificado. \\
	NasaPower &
	PRECTOTCORR 2010a2019JUN &
	Precipitação total corrigida registrada durante junho de cada ano de 2010 a 2019. O termo "corrigida" indica ajustes feitos para contabilizar possíveis erros ou viéses de medição. \\
	NasaPower &
	ALLSKY SFC LW DWN 2010a2019JUN &
	Radiação descendente de onda longa na superfície sob todas as condições de céu para junho ao longo dos anos de 2010 a 2019. Reflete a radiação de onda longa que alcança a superfície terrestre. \\
	NasaPower &
	ALLSKY SFC SW DWN 2010a2019JUN &
	Radiação descendente de onda curta na superfície sob todas as condições de céu durante junho para cada um dos anos de 2010 a 2019. \\
	NasaPower &
	alt 2010a2019JUN &
	Representa a altitude média do(s) local(is) de onde os dados de junho foram coletados ao longo desses anos. \\
	NasaPower &
	T2M 2010a2019MAR &
	Temperatura média a 2 metros acima do solo durante o mês de março para cada ano de 2010 a 2019. \\
	NasaPower &
	RH2M 2010a2019MAR &
	Umidade relativa média a 2 metros acima do solo durante março para o período especificado. \\
	NasaPower &
	WS2M 2010a2019MAR &
	Velocidade média do vento a 2 metros acima do solo medida durante março para os anos de 2010 a 2019. \\
	NasaPower &
	T2MDEW 2010a2019MAR &
	Temperatura média do ponto de orvalho a 2 metros durante março para os anos de 2010 a 2019. O ponto de orvalho é uma medida de quanto de umidade está no ar. \\
	NasaPower &
	T2M MAX 2010a2019MAR &
	Temperatura máxima média a 2 metros acima do solo durante março para os anos de 2010 a 2019. \\
	NasaPower &
	T2M MIN 2010a2019MAR &
	Temperatura mínima média a 2 metros acima do solo durante março durante o período especificado. \\
	NasaPower &
	PRECTOTCORR 2010a2019MAR &
	Precipitação total corrigida registrada durante março de cada ano de 2010 a 2019. O termo "corrigida" geralmente se refere a ajustes feitos nos dados para corrigir imprecisões ou viéses conhecidos. \\
	NasaPower &
	ALLSKY SFC LW DWN 2010a2019MAR &
	Radiação descendente de onda longa na superfície sob todas as condições de céu para março ao longo dos anos de 2010 a 2019. Reflete a radiação de onda longa que alcança a superfície terrestre. \\
	NasaPower &
	ALLSKY SFC SW DWN 2010a2019MAR &
	Radiação descendente de onda curta na superfície sob todas as condições de céu durante março para cada um dos anos de 2010 a 2019. \\
	NasaPower &
	alt 2010a2019MAR &
	Representa a altitude média do(s) local(is) de onde os dados de março foram coletados ao longo desses anos. \\
	NasaPower &
	T2M 2010a2019MAY &
	Temperatura média a 2 metros acima do solo durante o mês de maio para cada ano de 2010 a 2019. \\
	NasaPower &
	RH2M 2010a2019MAY &
	Umidade relativa média a 2 metros acima do solo durante maio para o período especificado. \\
	NasaPower &
	WS2M 2010a2019MAY &
	Velocidade média do vento a 2 metros acima do solo medida durante maio para os anos de 2010 a 2019. \\
	NasaPower &
	T2MDEW 2010a2019MAY &
	Temperatura média do ponto de orvalho a 2 metros durante maio para os anos de 2010 a 2019. A temperatura do ponto de orvalho é uma medida de umidade atmosférica. \\
	NasaPower &
	T2M MAX 2010a2019MAY &
	Temperatura máxima média a 2 metros acima do solo durante maio para os anos de 2010 a 2019. \\
	NasaPower &
	T2M MIN 2010a2019MAY &
	Temperatura mínima média a 2 metros acima do solo durante maio durante o período especificado. \\
	NasaPower &
	PRECTOTCORR 2010a2019MAY &
	Precipitação total corrigida registrada durante maio de cada ano de 2010 a 2019. As correções geralmente abordam imprecisões ou viéses de medição. \\
	NasaPower &
	ALLSKY SFC LW DWN 2010a2019MAY &
	Radiação descendente de onda longa na superfície sob todas as condições de céu para maio ao longo dos anos de 2010 a 2019. Reflete a radiação de onda longa que alcança a superfície terrestre. \\
	NasaPower &
	ALLSKY SFC SW DWN 2010a2019MAY &
	Radiação descendente de onda curta na superfície sob todas as condições de céu durante maio para cada um dos anos de 2010 a 2019. \\
	NasaPower &
	alt 2010a2019MAY &
	Representa a altitude média do(s) local(is) de onde os dados de maio foram coletados ao longo desses anos. \\
	NasaPower &
	T2M 2010a2019NOV &
	Temperatura média a 2 metros acima do solo durante o mês de novembro para cada ano de 2010 a 2019. \\
	NasaPower &
	RH2M 2010a2019NOV &
	Umidade relativa média a 2 metros acima do solo durante novembro para o período especificado. \\
	NasaPower &
	WS2M 2010a2019NOV &
	Velocidade média do vento a 2 metros acima do solo medida durante novembro para os anos de 2010 a 2019. \\
	NasaPower &
	T2MDEW 2010a2019NOV &
	Temperatura média do ponto de orvalho a 2 metros durante novembro para os anos de 2010 a 2019. O ponto de orvalho indica a quantidade de umidade no ar. \\
	NasaPower &
	T2M MAX 2010a2019NOV &
	Temperatura máxima média a 2 metros acima do solo durante novembro para os anos de 2010 a 2019. \\
	NasaPower &
	T2M MIN 2010a2019NOV &
	Temperatura mínima média a 2 metros acima do solo durante novembro durante o período especificado. \\
	NasaPower &
	PRECTOTCORR 2010a2019NOV &
	Precipitação total corrigida registrada durante novembro de cada ano de 2010 a 2019. O termo "corrigida" geralmente se refere a ajustes feitos nos dados para corrigir imprecisões ou viéses conhecidos. \\
	NasaPower &
	ALLSKY SFC LW DWN 2010a2019NOV &
	Radiação descendente de onda longa na superfície sob todas as condições de céu para novembro ao longo dos anos de 2010 a 2019. Reflete a radiação de onda longa que alcança a superfície terrestre. \\
	NasaPower &
	ALLSKY SFC SW DWN 2010a2019NOV &
	Radiação descendente de onda curta na superfície sob todas as condições de céu durante novembro para cada um dos anos de 2010 a 2019. \\
	NasaPower &
	alt 2010a2019NOV &
	Representa a altitude média do(s) local(is) de onde os dados de novembro foram coletados ao longo desses anos. \\
	NasaPower &
	T2M 2010a2019OCT &
	Temperatura média a 2 metros acima do solo durante o mês de outubro para cada ano de 2010 a 2019. \\
	NasaPower &
	RH2M 2010a2019OCT &
	Umidade relativa média a 2 metros acima do solo durante outubro para o período especificado. \\
	NasaPower &
	WS2M 2010a2019OCT &
	Velocidade média do vento a 2 metros acima do solo medida durante outubro para os anos de 2010 a 2019. \\
	NasaPower &
	T2MDEW 2010a2019OCT &
	Temperatura média do ponto de orvalho a 2 metros durante outubro para os anos de 2010 a 2019. O ponto de orvalho é uma medida de umidade atmosférica. \\
	NasaPower &
	T2M MAX 2010a2019OCT &
	Temperatura máxima média a 2 metros acima do solo durante outubro para os anos de 2010 a 2019. \\
	NasaPower &
	T2M MIN 2010a2019OCT &
	Temperatura mínima média a 2 metros acima do solo durante outubro durante o período especificado. \\
	NasaPower &
	PRECTOTCORR 2010a2019OCT &
	Precipitação total corrigida registrada durante outubro de cada ano de 2010 a 2019. Essa correção leva em conta imprecisões ou viéses de medição. \\
	NasaPower &
	ALLSKY SFC LW DWN 2010a2019OCT &
	Radiação descendente de onda longa na superfície sob todas as condições de céu para outubro ao longo dos anos de 2010 a 2019. Isso mede a radiação de onda longa que alcança a superfície terrestre. \\
	NasaPower &
	ALLSKY SFC SW DWN 2010a2019OCT &
	Radiação descendente de onda curta na superfície sob todas as condições de céu durante outubro para cada um dos anos de 2010 a 2019. \\
	NasaPower &
	alt 2010a2019OCT &
	Representa a altitude média do(s) local(is) de onde os dados de outubro foram coletados ao longo desses anos. \\
	NasaPower &
	T2M 2010a2019SEP &
	Temperatura média medida a 2 metros acima do solo durante o mês de setembro ao longo dos anos de 2010 a 2019. \\
	NasaPower &
	RH2M 2010a2019SEP &
	Umidade relativa média a 2 metros acima do solo durante setembro ao longo do período de 2010 a 2019. \\
	NasaPower &
	WS2M 2010a2019SEP &
	Velocidade média do vento a 2 metros acima do solo para setembro durante os anos especificados de 2010 a 2019. \\
	NasaPower &
	T2MDEW 2010a2019SEP &
	Temperatura média do ponto de orvalho a 2 metros acima do solo em setembro para os anos de 2010 a 2019. A temperatura do ponto de orvalho é uma medida da umidade atmosférica. \\
	NasaPower &
	T2M MAX 2010a2019SEP &
	Temperatura máxima média a 2 metros acima do solo durante setembro para cada ano de 2010 a 2019. \\
	NasaPower &
	T2M MIN 2010a2019SEP &
	Temperatura mínima média a 2 metros acima do solo durante setembro para os anos de 2010 a 2019. \\
	NasaPower &
	PRECTOTCORR 2010a2019SEP &
	Precipitação total registrada para setembro a cada ano de 2010 a 2019, corrigida para quaisquer viéses ou imprecisões conhecidos na medição. \\
	NasaPower &
	ALLSKY SFC LW DWN 2010a2019SEP &
	Quantidade média de radiação descendente de onda longa na superfície sob todas as condições de céu durante setembro para os anos especificados de 2010 a 2019, indicando a radiação de onda longa que alcança a superfície da Terra. \\
	NasaPower &
	ALLSKY SFC SW DWN 2010a2019SEP &
	Esta variável mede a média da radiação descendente de onda curta na superfície sob todas as condições de céu (como a luz solar) durante setembro para os anos de 2010 a 2019. \\
	NasaPower &
	alt 2010a2019SEP &
	Representa a altitude média do(s) local(is) de onde os dados de setembro foram coletados ao longo desses anos. \\ 
	\bottomrule
    \multicolumn{3}{@{}l}{\footnotesize\textit{Fonte: elaboração própria.}}\\
\end{longtable}


\end{document}

